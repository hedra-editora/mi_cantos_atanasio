\chapter[Para ler as palavras guarani]{Para ler as palavras\break guarani}
%\section{A escrita}

Na parte do livro escrita em língua guarani, o organizador e tradutor Izaque João utiliza a seguinte grafia: \textsc{a}, \textsc{e}, \textsc{i}, \textsc{o}, \textsc{u} e \textsc{y}, vogais de som aberto ou nasal; para identificar as vogais nasais, usa o símbolo \textit{til} (\textasciitilde{}) e, para identificar a consoante
glotal, que identifica uma pequena pausa entre vogais, utiliza a
\textit{apóstrofe} ('). A vogal \textsc{y} é uma alta central, que soa do céu da boca, com
som entre \textsc{i} e \textsc{u}. As letras consoantes presentes no alfabeto guarani aqui
adotadas são: \textsc{g}, \textsc{h}, \textsc{j}, \textsc{k}, \textsc{m}, \textsc{n}, \textsc{p}, \textsc{r}, \textsc{s}, \textsc{t}, \textsc{v} e \textsc{x}.

Também são usadas \textsc{nd}, \textsc{ng}, \textsc{nt}, \textsc{nh} e \textsc{mb}. A consoante \textsc{k} (\textit{pakova}, \textit{kuéra}, \textit{kyse}) é utilizada nos sons \textsc{c} (casa) e \textsc{qu} (quero) da língua portuguesa. A consoante \textsc{j} (\textit{jeguaka}, \textit{mbojegua}) é empregada nos fonemas de som \textsc{dj}. \textsc{h} (\textit{heta}, \textit{aháta}) é empregado nos
fonemas de som aspirado, como em \textsc{rr} do português falado em algumas
regiões do Brasil. Utilizo a consoante \textsc{ch} (\textit{chicha},
\textit{chiripa}) no lugar do \textsc{x}, embora ambas tenham o mesmo som, como o
do \textsc{x} (\textit{xícara}) em português. Ele usa também o \textsc{ñ}, no lugar de \textsc{nh} (\textit{ñande}, \textit{ñe'ẽ}, \textit{onhy}, \textit{oñomondo}), ambas com o mesmo som, como o do \textsc{ñ} em espanhol. Para o som \textsc{ss} ou \textsc{ç}, utiliza a consoante \textsc{s} (\textit{ohasa}, \textit{kyse}, \textit{guasu}).

Em conformidade com a grafia utilizada pelo Instituto Brasileiro de Geografia e Estatística (\textsc{ibge}) e outros órgãos públicos, excepcionalmente, utilizamos acento em Kaiowá.

%\textbf{Izaque João}
\pagebreak
\section{Como pronunciar}

As palavras que não têm acento são sempre oxítonas, ou seja, a tônica
fica na última sílaba: \textit{guasu} se diz ``guaçú''. Quando houver
acentos, a sílaba acentuada é a tônica: \textit{aháta} se diz
\textit{arráta}. O \textit{til} (\textasciitilde{}) não acentua a sílaba, a não ser
quando vem acompanhado do \textit{acento} (´). Para ter ideia dos sons, indicamos abaixo.

\bigskip

\begingroup%\footnotesize
\begin{tabular}{rl}
/a/ & parecido com as vogais em português\\
/e/ & parecido com as vogais em português\\
/i/ & parecido com as vogais em português\\
/o/ & parecido com as vogais em português\\
/u/ & parecido com as vogais em português\\
/y/ & entre \textit{i} e \textit{u}, mas com o centro da língua levantado\\
/ch/ & como em português: \textit{chá}, \textit{cheiro}\\
/j/ & como o \textit{j} em ``calça \textit{jeans}'' (\textit{dj})\\
/g/ & como o som em \textit{ga, go, gu }\\
/h/ & aspirado, como no \textit{rr} no português brasileiro, em \textit{carro}\\
/k/ & como o \textit{c} em \textit{casa} e \textit{escola}\\
/ñ/ & como o \textit{nh} em \textit{amanhã}\\
/r/ & sempre como o \textit{r} na palavra \textit{dourado}\\
/s/ & sempre como os \textit{ss} ou \textit{ç}\\
/v/ & como em português\\
/'/ & apóstrofe , como em \textit{ka'a}, pausa curta no som da voz\\
/d/ e /t/ & sempre com o som dessas letras em \textit{dado} ou \textit{tatu}\\
/mb/ & \textit{b} com som anasalado\\
/nd/ & \textit{d} de \textit{dado}, mas com som anasalado\\
/ng/ & \textit{g} de \textit{gato}, mas com som anasalado
\end{tabular}
\endgroup





