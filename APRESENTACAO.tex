\chapter*{Apresentação em português\smallskip\subtitulo{Teko Katu Maraneỹ}}
\addcontentsline{toc}{chapter}{Apresentação, \emph{por Atanásio Teixeira}}

\begin{flushright}
\textsc{atanásio teixeira}
\end{flushright}

\noindent{}\textls[5]{Por sermos habitantes desta terra, vou recordar e reviver a minha
infância passada, quando vivi com meu pai e minha mãe num lugar chamado
Guaviraty. Hoje é chamado de Vila Marques, e fica na região de fronteira.
Foi naquele lugar que, no passado, meu pai e minha mãe originaram a
minha vida. Também foi naquele lugar que, no passado, passei pela
cerimônia de furação do lábio (\emph{pepy}).\footnote{A cerimônia de furação do lábio, ou \emph{kunumi pepy}, é um rito que marca a passagem dos rapazes para a vida adulta.} Água de um lado, água do outro; nesse lugar, bem no meio, tinha casa de reza; nesse lugar, viviam
muitos de nossos parentes, faziam \emph{kunumi pepy}; nesse lugar havia
muitos líderes espirituais; lá todos dançavam muito. Daqui, ao passar,
já é Paraguai. Aqui existia mata, na água tinha muito peixe, na mata
vivia muito tatu e, quando era criança, eu comia muita carne de tatu. No
período em que eu nasci, o território era uma grande mata verde, tinha
muitos animais de caça para comer, nos rios havia muito peixe, no mato
tinha muita fruta nativa. Naquele lugar, também tinha muitos do nosso
povo; nossos parentes viviam unidos e dialogando.}

\textls[5]{No amplo espaço de Guaviraty, no início da minha infância, eu vivi, eu
andei muito com meu pai e com minha mãe. Meu pai se chamava Ava Jeguaka
Rendy e minha mãe chamava-se Mbo'y Rendy. Na minha fase de criança, no
lugar onde eu morava, não existia escola para nos ensinar a escrever no
papel. A minha escola ensinava só o que era nosso mesmo. Por meio das
palavras, meu pai e minha mãe me ensinavam, mostravam do nosso jeito
verdadeiro, como fazer mundéu e outros tipos de armadilha, para capturar
os bichos do mato. Lugar de ensinar era no mato, na beira de um rio ou
córrego, onde mostravam e ensinavam o jeito de fazer as coisas.}

Na minha casa, meu pai me ensinava para saber fazer peneira e cesta.
Também a cada manhã, na beira do fogo, o pai e a mãe ensinavam os
saberes para um viver verdadeiramente saudável no grupo. Com essa
aprendizagem, eu fixei em mim os verdadeiros saberes que nossos
antepassados haviam me ensinado. Naquele período, rapazes e moças
respeitavam e praticavam o nosso jeito de ser, não usavam a vida de
qualquer jeito, por isso a convivência acontecia em igualdade, viviam de
forma saudável, na vida em plenitude.

No meio do mato, quando andava com meu pai, ele me contava sobre as
madeiras de qualidades boas, e também mostrava as árvores indicadas para
espantar os donos das doenças. Também me mostrava remédios, trepadeiras
que ficam fixadas no galho das árvores e trepadeiras que ficam no tronco
das árvores. Ainda na andança com meu pai na mata, observei muitos
pássaros de diferentes espécies na mata, pássaros sagrados conhecidos
dos nossos antepassados, e já falava, são pássaros sagrados e pássaros
ruins. Meu pai, ao me ensinar, me dizia: ``escuta com sabedoria o canto
dos pássaros.'' Desse jeito ele falava para mim; meu pai me ensinava a
escutar com atenção os cantos de pássaro e interpretar o que diziam com
as suas vozes. Ficava escutando os cantos de pássaro, depois o meu pai
me ensinou um jeito de entender a comunicação de pássaro. Tem pássaro
cantando seu \emph{guahu} no meio da mata; outro que, através do canto,
anuncia algo ruim que existe na mata; também tem pássaros que cantam
juntos, de alegria.

Meu pai também mostrava várias ervas medicinais que nascem no brejo ou
na beira de córregos e rios. Existe muito remédio no brejo, de
diferentes espécies, para curar vários sintomas de doença ou prevenir
doenças maléficas, vários remédios específicos para homens ou mulheres,
e algumas espécies que podem ser consumidas por homens e mulheres.

Para conhecer os saberes verdadeiros, são necessários vários anos. Para
compreender cada canto ou para conhecer os muitos saberes antigos é
preciso muitas etapas de ensino, assim como nas escolas, para adquirir
conhecimentos profundos, tem que frequentar escola todos os dias. Só
assim nós vamos conseguir entender aqueles que falam sobre a vida plena.
Só assim nós vamos alcançar esse modo de ser. Aqui vocês certamente
vieram para ouvir e registrar sobre o sistema tradicional da comunidade
kaiowá: assim também, no passado, os grandes rezadores viviam buscando
sempre os conhecimentos verdadeiros.

\chapter*{Apresentação em guarani\smallskip\subtitulo{Teko Katu Maraneỹ}}

\begin{flushright}
\textsc{ava ñomoandyja}
\end{flushright}

\noindent{}Ñande yvypóra reko rehe, che mandu'áta ha ajevýta che mitã aikoramo
guare upe chesy ha cheru ndive amo Guaviratýpe. Ko'ãnga oñehenói Vila
Marques, upéa oĩme \textit{fronteira} rehe. Upépy tekohápy va'ekue hi'u ha
ha'i chembojehu va'ekue. Upépy chepepy va'ekue. Ko'ápy oĩme y, ápy ave
oĩme y, ndokova mbytépy oĩme ogusu va'ekue; upépy heta va'ekue oiko
ñande te'ýi, ojapo va'ekue kunumi pepy; upépy reta va'ekue oiko jekoha,
upépy ojeroky va'ekue hikuái. Ndo'ápy erehasa Paraguáima. Ko'ápy ka'aguy
va'ekue, reta pira ha tatu ave, upépy reta va'ekue vicho, che hetami
va'ekue ha'u tatu ro'o. Upéa ára chereñóirõ ramo guare, tekoha guasu
ka'aguypa va'ekue, reta oĩme mymba ka'aguy hi'upy, pira reta ýpy, reta
ave yva ka'aguy pygua. Upe Guaviratýpy reta va'ekue oĩme ñande re'ýi
kuéra, ñande kénte kuéry oñomongeta guasu.

Upe Guavyratýpy va'ekue reta ary aiko ha aguata va'ekue chesy ha cheru
ndive. Cheru héra va'ekue Ava Jeguaka Rendy ha chesy katu héra Mbo'y
Rendy. Chemitã ramo guare ndoikói va'ekue mbo'eróy umi ambue ohenóiva
escola porombo'éva kuatiarehe jeha'i haguã. Che mbo'eróy ñande mba'e ete
voi va'ekue. Ñe'ẽ rupi ochuka chévy che sy ha che ru che mbo'évy
va'ekue, ñande rekoetépy voi ochuka ajapo kuaa hag̃ua monde, opaichagua
ñuhã ijapopy. Ka'aguy rehegua oikóva roipyhy hag̃ua, che mbo'e renda ha'e
ka'aguy rehe, yrykóta rehe, upépy ochuka tembiapo rupi mba'éichapa
orojapo va'erã.

Ógapy che ru chembo'e umi yrupẽ rehe, ajaka ajapo kuaa hag̃ua rehe. Ha
reta ára tata kótare rehe ñe'ẽ rupive ombohasa chévy arandu teko resãi
marane'ỹ rehegua atyhápe. Upéa arandu che amoĩ chejehe arandu tee
ñemombe'u pyre. Upérupi va'ekue che aju akue, ñande reko rupi opamba'e
aikuaa. Teko ymave rupi mitãrusu, kuñatãi rusu omomba'e tee va'ekue
ñande reko, opáicha rei ndoiporúi okueko, upéagui ojehu va'ekue teko
joja, teko resãi, teko marangatu.

\textls[10]{Ka'aguy rupi katu cheru ndive aguata ramo, omombe'u va'ekue chévy yvyra
porã rehegua, yvyra mba'etirõ tĩha rehegua. Ochuka ave chévy pohã yvyra
rakã rehe oĩva ha yvyra rete rehe oĩva ave, upérupi che aikuaa reta pohã
ñande mba'e teéva. Cheru ndive joty ka'aguy rehe aguata jave ha reta voi
ahecha guyra mymba ka'aguy rupigua, umi guyra tee myamyrỹ kuéry omombe'u
pyre, umi guyra marangatu ha guyra rei ave. Cheru che mbo'évy chévy
he'i: ``Ejapysaka kuaa katu guyra ñe'ẽ rehe''. Upéicha he'i chévy, cheru
cherekombo'e ahendu kuaa hag̃ua guyra ñe'ẽ mba'épa he'i ijayvu rupi,
chembojapysaka umi guyra ñe'ẽ rehe, upéi cheru rekombo'e ikatuháicha
omombe'u, oĩva guyra ka'aguy rete rehe oguahu, oĩva katu guyra omombe'u
mba'eva'i ka'aguýre oĩméva, oĩva guyra katu ovy'águi oporahéi oñondive.}

Pohã yrykóta rehegua katu ochuka ave chévy cheru, pohã juvyy rupigua
reta avei oĩme héry opaichagua mba'asy pegua. Pohã ñahãna reta oĩme kuña
pegua ha oĩme ave kuimba'e pegua, ha ikatu ave kuña ha kuimba'e ho'u
peteĩ pohã.

Arandu jaikuaa hag̃ua heta ary jahasa, ñamboypy porã raguã arandu ha
mborahéi heta ary jajeheko mbo'e, upérupi ñande jaju ko'ãnga peve,
ha'ete ave ekuela, arandu porã jaikuaa hag̃ua ñambovia voi ekuela.
Upérupi mante jaikuaa porã teko marane'ỹ. Upérupi mante jaikuaa ñande
rekorã. Ko'ápe pende peju peraha hag̃ua arandu tee ñande reko rehegua:
árami voi jeiko va'ekue ymã rupi opa mba'e ñemboypy hag̃ua.