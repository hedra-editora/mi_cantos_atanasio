\chapter{Apresentação}

\section{Quem são os Kaiowa}

Os Kaiowa, junto com os Guarani, são hoje o segundo maior povo indígena
no país, com cerca de 50 mil pessoas, e habitam, em sua maioria, o sul
de Mato Grosso do Sul e regiões do Paraguai nessa fronteira com Brasil.
Historicamente, tratava-se de dois grupos distintos, falantes de duas
variantes da língua guarani - o kaiowa e o ñandeva. Ao longo da
colonização, porém, essas populações foram levadas a coabitar áreas no
sul do antigo estado de Mato Grosso. Hoje, conectadas por laços de
vizinhança e casamento, são chamadas nos documentos oficiais de
Guarani-Kaiowa, mas localmente preferem se autodenominar Kaiowa e
Guarani, ressaltando suas particularidades. Os Kaiowa são conhecidos no
Paraguai como Pai Tavyterã, e os Guarani, como Ava Katu Ete.

O que se conhece hoje como o estado do Mato Grosso do Sul corresponde
parcialmente à província da colonização espanhola conhecida como Itatim.
Durante o período colonial, entre os séculos XVI e XVII, os antigos
Guarani do Itatim fugiram do assédio dos colonos espanhóis, buscando
refúgio em missões jesuítas. Estas foram atacadas ao longo das décadas
de 1630 e 1640 pelos bandeirantes paulistas, que levaram como cativos
milhares de indígenas para São Paulo. Os grupos que conseguiram escapar
provavelmente buscaram as densas matas do atual sul de Mato Grosso do
Sul para refugiar-se até o século XIX, quando viajantes voltaram a
travar contato com essas populações. Após a Guerra da Tríplice Aliança,
ou Guerra do Paraguai (1864-70), que também atingiu diretamente esse
território, todo o atual sul do estado foi cedido para a exploração da
erva-mate nativa, promovida pela empresa Matte Larangeiras. Os indígenas
foram massivamente recrutados como trabalhadores nessa atividade
extrativista.

Entre 1917 e 1928, o Serviço de Proteção ao Índio (SPI) criou oito
reservas destinadas aos Kaiowa e Guarani. Nas décadas seguintes, a
colonização do sul de Mato Grosso do Sul avançou com apoio oficial até
que, nos anos 1970, expandem-se na região as lavouras de soja e cana.
Milhares de indígenas são expulsos das áreas que originalmente ocupavam
para serem confinados nas antigas reservas, que logo tornam-se áreas
superlotadas, pobres e violentas.

Ao mesmo tempo, emerge nas comunidades o movimento de luta pela
recuperação das terras tradicionais, chamado de Aty Guasu (\emph{grande
reunião}). Atanásio Teixeira é conhecido em Mato Grosso do Sul como
fundador e principal remanescente vivo entre os xamãs que impulsionaram
esse movimento, atualizando a reflexão kaiowa e guarani sobre o tema da
Terra sem Males, \emph{Yvy Maraneỹ}. Os territórios a serem
reconquistados são terra sagrada, a ser recuperada com auxílio dos
cantos xamânicos, é o que prega Ava Ñomoandyja.

\section{O que é um \emph{guahu}?}

Os \emph{guahu} são um dos diferentes tipos de cantos-rezas-danças dos
Kaiowa e Guarani, que podemos traduzir como ``cantos míticos'' (Tugny
\emph{et al}., 2016). Isso porque são, via de regra, associados a
narrativas a respeito do tempo da origem dos seres (\emph{ypy}) e dos
seus comportamentos. Todo \emph{guahu} ``tem história'', é o que se
costuma dizer entre os Kaiowa - e quase tudo que existe fala por meio de
seu \emph{guahu}, cantando como viveu no tempo da origem: há o canto da
flauta \emph{mimby}; o do chocalho \emph{mbaraka}; os \emph{guahu} de
aves, como as araras \emph{gua'a} e \emph{kaninde}; os \emph{guahu} dos
peixes e os dos animais da mata, \emph{mymba ka'aguy}; e até os cantos
\emph{guahu} de seres invisíveis, que os Kaiowa chamam de \emph{Pa'i
Re'i}.

Existem momentos específicos em que os \emph{guahu} são cantados: na
festa do milho branco - o \emph{avati kyry} - ou na festa da iniciação
dos meninos - o \emph{kunumi pepy} -, um \emph{guahu jary}, mestre dos
cantos \emph{guahu}, é quem canta, seguindo uma sequência determinada
pelo cantador, sem repetições. Mas os \emph{guahu} também podem ser
dançados em outros momentos de reunião e diversão, sem seguir uma
sequência específica.

Quando é feita em festas ou rituais como esses, a sequência dos cantos
\emph{guahu} costuma ser iniciada à noite, geralmente com o \emph{guahu}
em que Pa'i Kuara, o Sol, canta a sua história. Ao longo da noite, o
cantador pode dançar os \emph{guahu pyharegua}, \emph{guahu} noturnos,
mas existem \emph{guahu} que só podem ser cantados durante o dia: são os
\emph{guahu} \emph{arakuegua, guahu} diurnos, caso da maior parte dos
cantos selecionados por Ataná para esta edição. Os \emph{guahu} também
podem ser classificados como \emph{guahu yta}, \emph{guahungai} e
\emph{guahu ete}. O modo de combinar os diferentes tipos de \emph{guahu}
varia de acordo com a ocasião e a inspiração do cantador. Varia ainda a
forma como se dança: além de cantos e histórias do princípio, os
\emph{guahu} também são danças, feitas sempre em roda.

Nas festas mais importantes, os cantadores iniciam a dança em torno de
um balde de \emph{kaguῖ}, bebida fermentada de milho, sem dar as mãos,
batendo os pés no chão e marchando em sentido anti-horário. Nesses
momentos iniciais o \emph{guahu jary} carrega um grande arco,
\emph{guyrapa guasu}, nas mãos, e, se a mestra de \emph{guahu} for uma
mulher, em lugar do arco, ela deve portar uma haste de \emph{takuare'ẽ},
variedade tradicional de cana-de-açúcar. Depois de feitos dois ou três
\emph{guahu}, a sequência passa a ser cantada e dançada de mãos dadas,
sempre em roda.

As melodias dos \emph{guahu} de Atanásio Teixeira são cantadas em
registro grave, variando aproximadamente entre a nota Lá2 e o Fá3.
Outros cantadores podem empregar diferentes alturas e timbres, que
permitem identificar a pessoa e a região em que ela aprendeu seus
\emph{guahu}. O canto é acompanhado apenas pelas batidas dos pés dos
dançarinos e os versos são repetidos pelo \emph{guahu jary} duas ou três
vezes, ou mais, a depender da animação do grupo.

\section{Quem é Ava Ñomoandyja?}

Atanásio Teixeira ou Ava Ñomoandyja é um dos mais importantes
\emph{ñanderu} do povo Kaiowa em atividade. Nascido em 1922, Ataná é
chamado de \emph{ñamoῖ}, avô, por lideranças e rezadores de diferentes
comunidades kaiowa, pelos quais é reconhecido como mestre. É um dos
precursores dos \emph{Jeroky Guasu} (grandes danças) dos anos 1980, e do
movimento histórico pela recuperação dos territórios kaiowa e guarani em
Mato Grosso do Sul, a \emph{Aty Guasu} (grande reunião), sendo
reconhecido como um grande xamã também pelos Guarani.

\emph{Ñanderu}, ``nosso pai'', e \emph{ñandesy}, ``nossa mãe'' são
termos utilizados para se referir aos homens e mulheres do povo Kaiowa
que detêm conhecimentos xamânicos, evocando as denominações dadas aos
casais de prestígio das famílias extensas kaiowa e também usadas para se
referir às divindades nos discursos xamânicos e nos diálogos com elas,
por meio dos cantos-rezas. A divindade máxima dos Kaiowa, criador desta
terra, é Ñanderu Vusu, nome que pode ser traduzido como ``nosso grande
pai''. Os xamãs também costumam ser chamados, em Guarani, de
\emph{oporaeiva}, ``os que cantam'', \emph{ojerokyva}, ``os que
dançam'', \emph{oñembo'eva}, ``os que rezam'' e, no português indígena,
de \emph{rezadores} e \emph{caciques}. Outro epíteto aplicado aos
\emph{ñanderu} e \emph{ñandesy} é \emph{jekoha}, pilar ou esteio,
apontando que são o elemento fundamental para garantir a solidariedade e
a confiança coletiva, como se fossem o ``pilar da casa''.

O prestígio de Atanásio está, entre outros motivos, ligado ao fato de
dominar as mais variadas técnicas ligadas ao xamanismo kaiowa: os
\emph{ñembo'e}, fórmulas verbais de proteção pessoal ou coletiva; os
\emph{mborahei} e \emph{guahu}, cantos coletivos ligados aos rituais; os
diversos tipos de gestos conhecidos como \emph{jehovasa} - que podem ser
utilizados para influenciar as condições climáticas, desviando
tempestades, por exemplo; para curar doenças físicas e espirituais; para
garantir a sanidade das lavouras e colheitas etc.

Na parte inicial deste livro, \emph{Teko Katu Maraneỹ}, Ataná nos conta
parte de sua história de vida, lembrando de seu tempo de criança, quando
viveu com seus pais em Guaviraty, na região do atual município de Aral
Moreira (MS). Vivendo atualmente na Terra Indígena Limão Verde, no
município de Amambai (MS), Atanásio Teixeira passou ao longo de sua vida
por vários outros territórios de ocupação tradicional, chamados pelos
Kaiowa e Guarani de \emph{tekoha}, como Cerrito, Porto Lindo (Jakarey),
Jaguapire, Guasuty, Taquapery, Ñanderu Marangatu, Pyelito Kue. Em Limão
Verde, hoje é ele cuidado por uma de suas filhas, que também é
\emph{opuraeiva}, cantadora, e sua aprendiz.

Prestes a completar cem anos, Ava Ñomoandyja viveu o \emph{esparramo}
(\emph{sarambi}) das comunidades kaiowa e guarani e a devastação de seu
território em Mato Grosso do Sul, vendo um mundo de abundância e de
reciprocidade entre os grupos de famílias de seu povo praticamente
desaparecer. Por isso, ainda hoje, ele é um dos protagonistas da luta
pelo retorno a esses antigos territórios, empregando seus cantos-rezas e
suas palavras na \emph{retomada} dos \emph{tekoha}.

% \vfill
% \begin{flushright}
% Dourados--Porto Seguro--São Paulo, julho de 2021
% \end{flushright}