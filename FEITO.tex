\chapter{Como foi feito este livro}

Este livro apresenta uma pequena parte do imenso conjunto de
conhecimentos de um dos mais prestigiosos xamãs do povo Kaiowá: Ava
Ñomoandyja, Atanásio Teixeira. O livro foi produzido ao longo de seis
anos de encontros, diálogos, entrevistas e pesquisas com esse xamã, que
tiveram como ponto de partida a pesquisa coletiva Projeto de Sonoridades
(Prodocson), que integrou o Programa de Documentação de Línguas e
Culturas Indígenas (Progdoc) do Museu do Índio, entre 2013 e 2016.

\textls[-15]{Atanásio participou como pesquisador bolsista desse projeto de
abrangência nacional, coordenado pela etnomusicóloga Rosângela de Tugny.
O historiador Izaque João, organizador desta edição, e o antropólogo
Spensy Pimentel, co-organizador, foram os responsáveis pelo segmento
kaiowá e guarani da pesquisa. Depois, Izaque, atualmente
professor junto ao Curso Normal Médio Intercultural Ara Verá, em
Dourados (\textsc{ms}), manteve contato com Atanásio e deu continuidade aos
diálogos. A antropóloga Tatiane Klein juntou-se ao grupo em 2019, por
conta da pesquisa de doutorado sobre a circulação dos cantos kaiowá e
guarani, que realiza junto ao Centro de Estudos Ameríndios da
Universidade de São Paulo (\textsc{ce}st\textsc{a}--\textsc{usp}). Spensy, também pesquisador do \textsc{ce}st\textsc{a}--\textsc{usp}, é hoje professor da Universidade Federal do Sul da Bahia (\textsc{ufsb}).}\looseness=-1

Foi durante sua participação no Prodocson que Atanásio Teixeira
manifestou preocupação com o conjunto de cantos conhecido como ``os
\textit{guahu} dos animais primordiais'' (\textit{guyra guahu ha
mymba ka'aguy ayvu}). Nem todos os cantos entoados pelos xamãs podem ser
publicados ou explicados publicamente, por variados motivos. Quanto aos
cantos \textit{guahu}, porém, Atanásio manifesta-se particularmente
preocupado por sua preservação na memória coletiva dos Kaiowá e Guarani.
Grande parte das aldeias desse povo já não conserva hoje as condições
ambientais necessárias para que os jovens e crianças continuem mantendo
contato com os animais descritos nesses cantos. O desmatamento
generalizado e a consequente introdução da soja, da cana e da pecuária
mudaram radicalmente a região sul de Mato Grosso do Sul, anteriormente
conhecida pelos indígenas como uma área de \textit{Ka'aguyrusu}, ``mata
grande'', parte do bioma que os não indígenas chamam de Mata Atlântica.

%, que também estãodisponíveis para audição por meio dos códigos \textsc{qr}
Neste livro são apresentadas 26 histórias de aves e outros animais da
mata, acompanhados pelos cantos \textit{guahu} que cantam sua história
desde o princípio dos tempos. Esses \textit{guahu} fazem parte de um
conjunto maior de \textit{cantos-rezas-danças} dominado por Atanásio Teixeira, um
repertório registrado em áudio e vídeo ao longo da pesquisa. As
narrativas e explicações que acompanham os cantos foram elaboradas por
Izaque João, a partir de falas e orientações de Atanásio Teixeira ao
longo dos últimos seis anos. Os processos de seleção, transcrição e
tradução para esta edição bilíngue também foram feitos em diálogo com o
xamã, e as versões em português dos textos e cantos \textit{guahu} são um
exercício de aproximação a suas belas palavras.

A edição contou também com contribuições dos professores João Machado,
Gileandro Pedro, Veronice Lovato Rossato, Rosa Sebastiana Colman e
Arnulfo Morínigo, este último responsável pela revisão do texto em
língua guarani.
