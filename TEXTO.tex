\chapter{Como foi feito este livro}

Este livro apresenta uma pequena parte do imenso conjunto de
conhecimentos de um dos mais prestigiosos xamãs do povo Kaiowa: Ava
Ñomoandyja, Atanásio Teixeira. O livro foi produzido ao longo de seis
anos de encontros, diálogos, entrevistas e pesquisas com esse xamã, que
tiveram como ponto de partida a pesquisa coletiva Projeto de Sonoridades
(Prodocson), que integrou o Programa de Documentação de Línguas e
Culturas Indígenas (Progdoc) do Museu do Índio, entre 2013 e 2016.

Atanásio participou como pesquisador bolsista desse projeto de
abrangência nacional, coordenado pela etnomusicóloga Rosângela de Tugny.
O historiador Izaque João, coordenador desta edição, e o antropólogo
Spensy Pimentel, co-organizador, foram os responsáveis pelo segmento
kaiowa e guarani da pesquisa. Posteriormente, Izaque, atualmente
professor junto ao Curso Normal Médio Intercultural Ara Verá, em
Dourados (MS), manteve contato com Atanásio, dando continuidade aos
diálogos. A antropóloga Tatiane Klein juntou-se ao grupo em 2019, por
conta da pesquisa de doutorado sobre a circulação dos cantos kaiowa e
guarani, que realiza junto ao Centro de Estudos Ameríndios da
Universidade de São Paulo (CEstA-USP). Spensy, também pesquisador do
CEstA-USP, é hoje professor da Universidade Federal do Sul da Bahia
(UFSB).

Foi durante sua participação no Prodocson que Atanásio Teixeira
manifestou a preocupação com o conjunto de cantos conhecido como ``os
\emph{guahu} dos animais primordiais'' (\emph{guyra guahu} \emph{ha
mymba ka'aguy ayvu}). Nem todos os cantos entoados pelos xamãs podem ser
publicados ou explicados publicamente, por variados motivos. Quanto aos
cantos \emph{guahu}, porém, Atanásio manifesta-se particularmente
preocupado por sua preservação na memória coletiva dos Kaiowa e Guarani.
Grande parte das aldeias desse povo já não conserva hoje as condições
ambientais necessárias para que os jovens e crianças continuem mantendo
contato com os animais descritos nesses cantos. O desmatamento
generalizado e a consequente introdução da soja, da cana e da pecuária
mudaram radicalmente a região sul de Mato Grosso do Sul, anteriormente
conhecida pelos indígenas como uma área de \emph{Ka'aguyrusu} (mata
grande), parte do bioma que os não indígenas chamam de Mata Atlântica.

Neste livro são apresentadas 26 histórias de aves e outros animais da
mata, acompanhados pelos cantos \emph{guahu} que cantam sua história
desde o princípio dos tempos. Esses \emph{guahu}, que também estão
disponíveis para audição por meio dos códigos QR, fazem parte de um
conjunto maior de cantos-rezas-danças dominado por Atanásio Teixeira, um
repertório registrado em áudio e vídeo ao longo da pesquisa. As
narrativas e explicações que acompanham os cantos foram elaboradas por
Izaque João, a partir de falas e orientações de Atanásio Teixeira ao
longo dos últimos seis anos. Os processos de seleção, transcrição e
tradução para esta edição bilíngue também foram feitos em diálogo com o
xamã e as versões em português dos textos e cantos \emph{guahu} são um
exercício de aproximação a suas belas palavras.

A edição contou também com contribuições dos professores João Machado,
Gileandro Pedro, Veronice Lovato Rossato, Rosa Sebastiana Colman e
Arnulfo Morínigo, este último responsável pela revisão do texto em
língua guarani.

\chapter{Quem é Ava Ñomoandyja?}

Atanásio Teixeira ou Ava Ñomoandyja é um dos mais importantes
\emph{ñanderu} do povo Kaiowa em atividade. Nascido em 1922, Ataná é
chamado de \emph{ñamoῖ}, avô, por lideranças e rezadores de diferentes
comunidades kaiowa, pelos quais é reconhecido como mestre. É um dos
precursores dos \emph{Jeroky Guasu} (grandes danças) dos anos 1980, e do
movimento histórico pela recuperação dos territórios kaiowa e guarani em
Mato Grosso do Sul, a \emph{Aty Guasu} (grande reunião), sendo
reconhecido como um grande xamã também pelos Guarani.

\emph{Ñanderu}, ``nosso pai'', e \emph{ñandesy}, ``nossa mãe'' são
termos utilizados para se referir aos homens e mulheres do povo Kaiowa
que detêm conhecimentos xamânicos, evocando as denominações dadas aos
casais de prestígio das famílias extensas kaiowa e também usadas para se
referir às divindades nos discursos xamânicos e nos diálogos com elas,
por meio dos cantos-rezas. A divindade máxima dos Kaiowa, criador desta
terra, é Ñanderu Vusu, nome que pode ser traduzido como ``nosso grande
pai''. Os xamãs também costumam ser chamados, em Guarani, de
\emph{oporaeiva}, ``os que cantam'', \emph{ojerokyva}, ``os que
dançam'', \emph{oñembo'eva}, ``os que rezam'' e, no português indígena,
de \emph{rezadores} e \emph{caciques}. Outro epíteto aplicado aos
\emph{ñanderu} e \emph{ñandesy} é \emph{jekoha}, pilar ou esteio,
apontando que são o elemento fundamental para garantir a solidariedade e
a confiança coletiva, como se fossem o ``pilar da casa''.

O prestígio de Atanásio está, entre outros motivos, ligado ao fato de
dominar as mais variadas técnicas ligadas ao xamanismo kaiowa: os
\emph{ñembo'e}, fórmulas verbais de proteção pessoal ou coletiva; os
\emph{mborahei} e \emph{guahu}, cantos coletivos ligados aos rituais; os
diversos tipos de gestos conhecidos como \emph{jehovasa} - que podem ser
utilizados para influenciar as condições climáticas, desviando
tempestades, por exemplo; para curar doenças físicas e espirituais; para
garantir a sanidade das lavouras e colheitas etc.

Na parte inicial deste livro, \emph{Teko Katu Maraneỹ}, Ataná nos conta
parte de sua história de vida, lembrando de seu tempo de criança, quando
viveu com seus pais em Guaviraty, na região do atual município de Aral
Moreira (MS). Vivendo atualmente na Terra Indígena Limão Verde, no
município de Amambai (MS), Atanásio Teixeira passou ao longo de sua vida
por vários outros territórios de ocupação tradicional, chamados pelos
Kaiowa e Guarani de \emph{tekoha}, como Cerrito, Porto Lindo (Jakarey),
Jaguapire, Guasuty, Taquapery, Ñanderu Marangatu, Pyelito Kue. Em Limão
Verde, hoje é ele cuidado por uma de suas filhas, que também é
\emph{opuraeiva}, cantadora, e sua aprendiz.

Prestes a completar cem anos, Ava Ñomoandyja viveu o \emph{esparramo}
(\emph{sarambi}) das comunidades kaiowa e guarani e a devastação de seu
território em Mato Grosso do Sul, vendo um mundo de abundância e de
reciprocidade entre os grupos de famílias de seu povo praticamente
desaparecer. Por isso, ainda hoje, ele é um dos protagonistas da luta
pelo retorno a esses antigos territórios, empregando seus cantos-rezas e
suas palavras na \emph{retomada} dos \emph{tekoha}.

\emph{Dourados-Porto Seguro-São Paulo, julho de 2021.}

\chapter{O que é um \emph{guahu}?}

Os \emph{guahu} são um dos diferentes tipos de cantos-rezas-danças dos
Kaiowa e Guarani, que podemos traduzir como ``cantos míticos'' (Tugny
\emph{et al}., 2016). Isso porque são, via de regra, associados a
narrativas a respeito do tempo da origem dos seres (\emph{ypy}) e dos
seus comportamentos. Todo \emph{guahu} ``tem história'', é o que se
costuma dizer entre os Kaiowa - e quase tudo que existe fala por meio de
seu \emph{guahu}, cantando como viveu no tempo da origem: há o canto da
flauta \emph{mimby}; o do chocalho \emph{mbaraka}; os \emph{guahu} de
aves, como as araras \emph{gua'a} e \emph{kaninde}; os \emph{guahu} dos
peixes e os dos animais da mata, \emph{mymba ka'aguy}; e até os cantos
\emph{guahu} de seres invisíveis, que os Kaiowa chamam de \emph{Pa'i
Re'i}.

Existem momentos específicos em que os \emph{guahu} são cantados: na
festa do milho branco - o \emph{avati kyry} - ou na festa da iniciação
dos meninos - o \emph{kunumi pepy} -, um \emph{guahu jary}, mestre dos
cantos \emph{guahu}, é quem canta, seguindo uma sequência determinada
pelo cantador, sem repetições. Mas os \emph{guahu} também podem ser
dançados em outros momentos de reunião e diversão, sem seguir uma
sequência específica.

Quando é feita em festas ou rituais como esses, a sequência dos cantos
\emph{guahu} costuma ser iniciada à noite, geralmente com o \emph{guahu}
em que Pa'i Kuara, o Sol, canta a sua história. Ao longo da noite, o
cantador pode dançar os \emph{guahu pyharegua}, \emph{guahu} noturnos,
mas existem \emph{guahu} que só podem ser cantados durante o dia: são os
\emph{guahu} \emph{arakuegua, guahu} diurnos, caso da maior parte dos
cantos selecionados por Ataná para esta edição. Os \emph{guahu} também
podem ser classificados como \emph{guahu yta}, \emph{guahungai} e
\emph{guahu ete}. O modo de combinar os diferentes tipos de \emph{guahu}
varia de acordo com a ocasião e a inspiração do cantador. Varia ainda a
forma como se dança: além de cantos e histórias do princípio, os
\emph{guahu} também são danças, feitas sempre em roda.

Nas festas mais importantes, os cantadores iniciam a dança em torno de
um balde de \emph{kaguῖ}, bebida fermentada de milho, sem dar as mãos,
batendo os pés no chão e marchando em sentido anti-horário. Nesses
momentos iniciais o \emph{guahu jary} carrega um grande arco,
\emph{guyrapa guasu}, nas mãos, e, se a mestra de \emph{guahu} for uma
mulher, em lugar do arco, ela deve portar uma haste de \emph{takuare'ẽ},
variedade tradicional de cana-de-açúcar. Depois de feitos dois ou três
\emph{guahu}, a sequência passa a ser cantada e dançada de mãos dadas,
sempre em roda.

As melodias dos \emph{guahu} de Atanásio Teixeira são cantadas em
registro grave, variando aproximadamente entre a nota Lá2 e o Fá3.
Outros cantadores podem empregar diferentes alturas e timbres, que
permitem identificar a pessoa e a região em que ela aprendeu seus
\emph{guahu}. O canto é acompanhado apenas pelas batidas dos pés dos
dançarinos e os versos são repetidos pelo \emph{guahu jary} duas ou três
vezes, ou mais, a depender da animação do grupo.

\chapter{Quem são os Kaiowa}

Os Kaiowa, junto com os Guarani, são hoje o segundo maior povo indígena
no país, com cerca de 50 mil pessoas, e habitam, em sua maioria, o sul
de Mato Grosso do Sul e regiões do Paraguai nessa fronteira com Brasil.
Historicamente, tratava-se de dois grupos distintos, falantes de duas
variantes da língua guarani - o kaiowa e o ñandeva. Ao longo da
colonização, porém, essas populações foram levadas a coabitar áreas no
sul do antigo estado de Mato Grosso. Hoje, conectadas por laços de
vizinhança e casamento, são chamadas nos documentos oficiais de
Guarani-Kaiowa, mas localmente preferem se autodenominar Kaiowa e
Guarani, ressaltando suas particularidades. Os Kaiowa são conhecidos no
Paraguai como Pai Tavyterã, e os Guarani, como Ava Katu Ete.

O que se conhece hoje como o estado do Mato Grosso do Sul corresponde
parcialmente à província da colonização espanhola conhecida como Itatim.
Durante o período colonial, entre os séculos XVI e XVII, os antigos
Guarani do Itatim fugiram do assédio dos colonos espanhóis, buscando
refúgio em missões jesuítas. Estas foram atacadas ao longo das décadas
de 1630 e 1640 pelos bandeirantes paulistas, que levaram como cativos
milhares de indígenas para São Paulo. Os grupos que conseguiram escapar
provavelmente buscaram as densas matas do atual sul de Mato Grosso do
Sul para refugiar-se até o século XIX, quando viajantes voltaram a
travar contato com essas populações. Após a Guerra da Tríplice Aliança,
ou Guerra do Paraguai (1864-70), que também atingiu diretamente esse
território, todo o atual sul do estado foi cedido para a exploração da
erva-mate nativa, promovida pela empresa Matte Larangeiras. Os indígenas
foram massivamente recrutados como trabalhadores nessa atividade
extrativista.

Entre 1917 e 1928, o Serviço de Proteção ao Índio (SPI) criou oito
reservas destinadas aos Kaiowa e Guarani. Nas décadas seguintes, a
colonização do sul de Mato Grosso do Sul avançou com apoio oficial até
que, nos anos 1970, expandem-se na região as lavouras de soja e cana.
Milhares de indígenas são expulsos das áreas que originalmente ocupavam
para serem confinados nas antigas reservas, que logo tornam-se áreas
superlotadas, pobres e violentas.

Ao mesmo tempo, emerge nas comunidades o movimento de luta pela
recuperação das terras tradicionais, chamado de Aty Guasu (\emph{grande
reunião}). Atanásio Teixeira é conhecido em Mato Grosso do Sul como
fundador e principal remanescente vivo entre os xamãs que impulsionaram
esse movimento, atualizando a reflexão kaiowa e guarani sobre o tema da
Terra sem Males, \emph{Yvy Maraneỹ}. Os territórios a serem
reconquistados são terra sagrada, a ser recuperada com auxílio dos
cantos xamânicos, é o que prega Ava Ñomoandyja.

\chapter{A escrita das palavras em Guarani}

Na escrita na língua guarani eu utilizo a seguinte grafia: A, E, I, O, U
e Y, vogais de som aberto ou nasal; para identificar as vogais nasais,
uso o símbolo til (\textasciitilde{}) e, para identificar a consoante
glotal, que identifica uma pequena pausa entre vogais, utilizo a
apóstrofe ('). A vogal Y é uma alta central, que soa do céu da boca, com
som entre I e U. As letras consoantes presentes no alfabeto guarani aqui
adotadas são: G, H, J, K, M, N, P, R, S, T, V e X.

Também são usadas ND, NG, NT, NH e MB. A consoante K (\emph{pakova},
\emph{kuéra, kyse}) é utilizada nos sons C (casa) e QU (quero) da língua
portuguesa. A consoante J (\emph{jeguaka}, \emph{mbojegua}) é empregada
nos fonemas de som ``DJ''. H (\emph{heta}, \emph{aháta}) é empregado nos
fonemas de som aspirado, como em ``RR'' do português falado em algumas
regiões do Brasil. Utilizo a consoante CH (\emph{chicha},
\emph{chiripa}) no lugar do X, embora ambas tenham o mesmo som, como o
do X (\emph{xícara}) em português. Uso Ñ, no lugar de NH (\emph{ñande,
ñe'ẽ, onhy, oñomondo}), ambas com o mesmo som, como o do Ñ em Espanhol.
Para o som ``SS'' ou ``Ç'', utilizo a consoante S (\emph{ohasa, kyse,
guasu).}

%\textbf{Izaque João}

\chapter{Como pronunciar as palavras guarani}

As palavras que não têm acento são sempre oxítonas, ou seja, a tônica
fica na última sílaba: \emph{guasu} se diz ``guaçú''. Quando houver
acentos, a sílaba acentuada é a tônica: \emph{aháta} se diz
\emph{arráta}. O til (\textasciitilde{}) não acentua a sílaba, a não ser
quando vem acompanhado do acento ( ´ ).

Para ter ideia dos sons da língua, indicamos abaixo.

\begin{itemize}
\item[/a/] com o som parecido com as vogais em português
\item[/e/] com o som parecido com as vogais em português
\item[/i/] com o som parecido com as vogais em português
\item[/o/] com o som parecido com as vogais em português
\item[/u/] com o som parecido com as vogais em português

\item[/y/] como um som entre \emph{i} e \emph{u}, mas com a parte central da
língua um pouco levantada dentro da boca

\item[/ch/] como em português: \emph{chá}, \emph{cheiro}

\item[/j/] como o \emph{j} em ``calça \emph{jeans}'' (\emph{dj})

\item[/g/] sempre como o som em \emph{ga, go, gu }

\item[/h/] aspirado, como no \emph{rr} no português brasileiro, em \emph{carro}

\item[/k/] como o \emph{c} em \emph{casa} e \emph{escola}

\item[/ñ/] como o \emph{nh} em \emph{amanhã}

\item[/r/] sempre como o \emph{r} na palavra \emph{dourado}

\item[/s/] sempre como os \emph{ss} ou \emph{ç}

\item[/v/] como em português

\item[/'/] (apóstrofe, como em \emph{ka'a}) é uma pausa curta no som da voz

\item[/d/] e /t/ sempre com o som dessas letras em \emph{dado} ou \emph{tatu}

\item[/mb/] \emph{b} com som anasalado

\item[/nd/] \emph{d} de \emph{dado,} mas com som anasalado

\item[/ng/] \emph{g} de \emph{gato}, mas com som anasalado
\end{itemize}

\part{Teko katu maraneỹ -- Palavras de Atanásio Teixeira}

\chapter{Tekoha Guaviraty}

Ñande yvypóra reko rehe, che mandu'áta ha ajevýta che mitã aikoramo
guare upe chesy ha cheru ndive amo Guaviratýpe. Ko'ãnga oñehenói Vila
Marques, upéa oĩme \emph{fronteira} rehe. Upépy tekohápy va'ekue hi'u ha
ha'i chembojehu va'ekue. Upépy chepepy va'ekue. Ko'ápy oĩme y, ápy ave
oĩme y, ndokova mbytépy oĩme ogusu va'ekue; upépy heta va'ekue oiko
ñande te'ýi, ojapo va'ekue kunumi pepy; upépy reta va'ekue oiko jekoha,
upépy ojeroky va'ekue hikuái. Ndo'ápy erehasa Paraguáima. Ko'ápy ka'aguy
va'ekue, reta pira ha tatu ave, upépy reta va'ekue vicho, che hetami
va'ekue ha'u tatu ro'o. Upéa ára chereñóirõ ramo guare, tekoha guasu
ka'aguypa va'ekue, reta oĩme mymba ka'aguy hi'upy, pira reta ýpy, reta
ave yva ka'aguy pygua. Upe Guaviratýpy reta va'ekue oĩme ñande re'ýi
kuéra, ñande kénte kuéry oñomongeta guasu.

\begin{quote}
Por sermos habitantes desta terra, vou recordar e reviver a minha
infância passada, quando vivi com meu pai e minha mãe num lugar chamado
Guaviraty. Hoje é chamado de Vila Marques e fica na região de fronteira.
Foi naquele lugar que, no passado, meu pai e minha mãe originaram a
minha vida. Também naquele lugar foi que, no passado, passei pela
cerimônia de furação do lábio (\emph{pepy})\footnote{A cerimônia de
  furação do lábio, ou \emph{kunumi pepy}, é um rito que marca a
  passagem dos rapazes para a vida adulta.}. Água de um lado, água do
outro; nesse lugar, bem no meio, tinha casa de reza; nesse lugar, viviam
muitos de nossos parentes, faziam \emph{kunumi pepy}; nesse lugar havia
muitos líderes espirituais; lá todos dançavam muito. Daqui, ao passar,
já é Paraguai. Aqui existia mata, na água tinha muito peixe, na mata
vivia muito tatu e, quando era criança, eu comia muita carne de tatu. No
período em que eu nasci, o território era uma grande mata verde, tinha
muitos animais de caça para comer, nos rios havia muito peixe, no mato
tinha muita fruta nativa. Naquele lugar, também tinha muitos do nosso
povo; nossos parentes viviam unidos e dialogando.
\end{quote}

Upe Guavyratýpy va'ekue reta ary aiko ha aguata va'ekue chesy ha cheru
ndive. Cheru héra va'ekue Ava Jeguaka Rendy ha chesy katu héra Mbo'y
Rendy. Chemitã ramo guare ndoikói va'ekue mbo'eróy umi ambue ohenóiva
escola porombo'éva kuatiarehe jeha'i hag̃ua. Che mbo'eróy ñande mba'e ete
voi va'ekue. Ñe'ẽ rupi ochuka chévy che sy ha che ru che mbo'évy
va'ekue, ñande rekoetépy voi ochuka ajapo kuaa hag̃ua monde, opaichagua
ñuhã ijapopy. Ka'aguy rehegua oikóva roipyhy hag̃ua, che mbo'e renda ha'e
ka'aguy rehe, yrykóta rehe, upépy ochuka tembiapo rupi mba'éichapa
orojapo va'erã.

\begin{quote}
No amplo espaço de Guaviraty, no início da minha infância, eu vivi, eu
andei muito com meu pai e com minha mãe. Meu pai se chamava Ava Jeguaka
Rendy e minha mãe chamava-se Mbo'y Rendy. Na minha fase de criança, no
lugar onde eu morava, não existia escola para nos ensinar a escrever no
papel. A minha escola ensinava só o que era nosso mesmo. Por meio das
palavras, meu pai e minha mãe me ensinavam, mostravam do nosso jeito
verdadeiro, como fazer mundéu e outros tipos de armadilha, para capturar
os bichos do mato. Lugar de ensinar era no mato, na beira de um rio ou
córrego, onde mostravam e ensinavam o jeito de fazer as coisas.
\end{quote}

Ógapy che ru chembo'e umi yrupẽ rehe, ajaka ajapo kuaa hag̃ua rehe. Ha
reta ára tata kótare rehe ñe'ẽ rupive ombohasa chévy arandu teko resãi
marane'ỹ rehegua atyhápe. Upéa arandu che amoĩ chejehe arandu tee
ñemombe'u pyre. Upérupi va'ekue che aju akue, ñande reko rupi opamba'e
aikuaa. Teko ymave rupi mitãrusu, kuñatãi rusu omomba'e tee va'ekue
ñande reko, opáicha rei ndoiporúi okueko, upéagui ojehu va'ekue teko
joja, teko resãi, teko marangatu.

\begin{quote}
Na minha casa, meu pai me ensinava para saber fazer peneira e cesta.
Também a cada manhã, na beira do fogo, o pai e a mãe ensinavam os
saberes para um viver verdadeiramente saudável no grupo. Com essa
aprendizagem, eu fixei em mim os verdadeiros saberes que nossos
antepassados haviam me ensinado. Naquele período, rapazes e moças
respeitavam e praticavam o nosso jeito de ser, não usavam a vida de
qualquer jeito, por isso a convivência acontecia em igualdade, viviam de
forma saudável, na vida em plenitude.
\end{quote}

Ka'aguy rupi katu cheru ndive aguata ramo, omombe'u va'ekue chévy yvyra
porã rehegua, yvyra mba'etirõ tĩha rehegua. Ochuka ave chévy pohã yvyra
rakã rehe oĩva ha yvyra rete rehe oĩva ave, upérupi che aikuaa reta pohã
ñande mba'e teéva. Cheru ndive joty ka'aguy rehe aguata jave ha reta voi
ahecha guyra mymba ka'aguy rupigua, umi guyra tee myamyrỹ kuéry omombe'u
pyre, umi guyra marangatu ha guyra rei ave. Cheru che mbo'évy chévy
he'i: ``Ejapysaka kuaa katu guyra ñe'ẽ rehe''. Upéicha he'i chévy, cheru
cherekombo'e ahendu kuaa hag̃ua guyra ñe'ẽ mba'épa he'i ijayvu rupi,
chembojapysaka umi guyra ñe'ẽ rehe, upéi cheru rekombo'e ikatuháicha
omombe'u, oĩva guyra ka'aguy rete rehe oguahu, oĩva katu guyra omombe'u
mba'eva'i ka'aguýre oĩméva, oĩva guyra katu ovy'águi oporahéi oñondive.

\begin{quote}
No meio do mato, quando andava com meu pai, ele me contava sobre as
madeiras de qualidades boas, e também mostrava as árvores indicadas para
espantar os donos das doenças. Também me mostrava remédios, trepadeiras
que ficam fixadas no galho das árvores e trepadeiras que ficam no tronco
das árvores. Ainda na andança com meu pai na mata, observei muitos
pássaros de diferentes espécies na mata, pássaros sagrados conhecidos
dos nossos antepassados, e já falava, são pássaros sagrados e pássaros
ruins. Meu pai, ao me ensinar, me dizia: ``Escuta com sabedoria o canto
dos pássaros''. Desse jeito ele falava para mim; meu pai me ensinava a
escutar com atenção os cantos de pássaro e interpretar o que diziam com
as suas vozes. Ficava escutando os cantos de pássaro, depois o meu pai
me ensinou um jeito de entender a comunicação de pássaro. Tem pássaro
cantando seu \emph{guahu} no meio da mata; outro que, através do canto,
anuncia algo ruim que existe na mata; também tem pássaros que cantam
juntos, de alegria.
\end{quote}

Pohã yrykóta rehegua katu ochuka ave chévy cheru, pohã juvyy rupigua
reta avei oĩme héry opaichagua mba'asy pegua. Pohã ñahãna reta oĩme kuña
pegua ha oĩme ave kuimba'e pegua, ha ikatu ave kuña ha kuimba'e ho'u
peteĩ pohã.

\begin{quote}
Meu pai também mostrava várias ervas medicinais que nascem no brejo ou
na beira de córregos e rios. Existe muito remédio no brejo, de
diferentes espécies, para curar vários sintomas de doença ou prevenir
doenças maléficas, vários remédios específicos para homens ou mulheres,
e algumas espécies que podem ser consumidas por homens e mulheres.
\end{quote}

Arandu jaikuaa hag̃ua heta ary jahasa, ñamboypy porã raguã arandu ha
mborahéi heta ary jajeheko mbo'e, upérupi ñande jaju ko'ãnga peve,
ha'ete ave ekuela, arandu porã jaikuaa hag̃ua ñambovia voi ekuela.
Upérupi mante jaikuaa porã teko marane'ỹ. Upérupi mante jaikuaa ñande
rekorã. Ko'ápe pende peju peraha hag̃ua arandu tee ñande reko rehegua:
árami voi jeiko va'ekue ymã rupi opa mba'e ñemboypy hag̃ua.

\begin{quote}
Para conhecer os saberes verdadeiros, são necessários vários anos. Para
compreender cada canto ou para conhecer os muitos saberes antigos é
preciso muitas etapas de ensino, assim como nas escolas, para adquirir
conhecimentos profundos, tem que frequentar escola todos os dias. Só
assim nós vamos conseguir entender aqueles que falam sobre a vida plena.
Só assim nós vamos alcançar esse modo de ser. Aqui vocês certamente
vieram para ouvir e registrar sobre o sistema tradicional da comunidade
kaiowa: assim também, no passado, os grandes rezadores viviam buscando
sempre os conhecimentos verdadeiros.
\end{quote}

\part{Parte II}

\chapterspecial{Cantos dos animais primordiais}{Guyra guahu ha mymba ka'aguy ayvu}{}

No Princípio, chamado pelos Kaiowa e Guarani de \emph{Áraypy}, todas as
aves e animais da mata conversavam uns com os outros, eram considerados
humanos. Um dia, seguindo os irmãos Pa'i Kuara e Jasy, Sol e Lua, na
travessia de um rio, as aves e animais da mata foram derrubados na água
pelo irmão mais novo, Jasy, e foram transformados: nunca mais voltaram a
se entender. Até hoje, cada um deles conta suas origens e seu modo de
ser por meio de um canto \emph{guahu}.

\chapter{Jarara rehegua ñemombe'u / O canto da Jararaca}

Mbói mba'e meg̃ua voi, upéagui Jararái jari oipapa jokupe araka'e ogueko
rehegua iguahu yvypórape, ha'e omombe'u oguahu rupi ogupi'a hete
rehegua.

\begin{quote}
A Jararaca é considerada uma criatura monstruosa, por isso essa cobra
possui seu dono. Esse dono da cobra, ao contar sobre o seu jeito de ser,
escondia por completo a sua maldade, falando apenas sobre o ambiente de
sua reprodução e os ovos que estão na extensão de seu corpo.
\end{quote}

Añetéma, Mbói Jarara oguereko ogupi'a ipuku jave voi oñesyrũ rete rehe;
upéa va'e Jararái omombe'u oguahu rupi. Mbói Jari ypy niko oñeha'ã
mbarete voi araka'e omoherakuã rag̃ua ojehegua iporãeteha, ndaikatúi
mante oipapa ojehegua rekotee rupi, upéagui Mbói Jari omombe'u jokupe ha
upéicha araka'e he'i: ``Che ahejáta ave che guahu yvypórape ko'ẽ ko'ẽ
rupi jeguahu hápy oñeha'ã rag̃ua''. He'i mbói jary, upéicha oguahu Jarara
ypy omboyvypóry araka'e. Upéagui jeko Jarara guahu oñeha'ã ko'ẽ rupi
mante, oipapa hesegua hupi'a ipuku jave oĩmeha reterehe. Kóa guahu
oipapáva Jarara rehegua iporã jaikuaa ñamombe'u joapýri pýri rag̃ua ñande
kénte kuérype. Jarara guahúko yvy pype he'i:

\begin{quote}
É verdade que a Jararaca possui vários ovos no seu interior; isso é
contado por meio de seu canto \emph{guahu}. Mbói Jari, no princípio de
sua existência, se esforçou muito para espalhar por toda parte a sua
narrativa de enganosa benevolência, tentando esconder a qualquer custo a
sua maldade. Então ele disse assim: ``Eu também quero deixar para os
seres da terra o meu canto \emph{guahu}, a ser cantado ao amanhecer''.
Assim disse o dono da cobra, Mbói Jari, e desde então ele deixou o seu
\emph{guahu} aqui na Terra. Por isso o \emph{guahu} da cobra Jararaca só
é cantado no momento do amanhecer; ele fala sobre o ambiente de
reprodução e os ovos que a cobra carrega no seu interior. Essa narrativa
é repassada de geração em geração aos nossos parentes. O \emph{guahu} da
Jararaca é assim:
\end{quote}

\begin{table}[]
\begin{tabular}{rl}
Jarara ri jari                       & Dono da Jararaca                               \\
Yvyra pykue puku\footnotemark aja katu nde rupi'a & Raiz comprida da árvore, lá ela bota seus ovos \\
Jarara ri jari                       & Dono da Jararaca                               \\
Yvyra pykue puku aja katu nde rupi'a & Raiz comprida da árvore, lá ela bota seus ovos
\end{tabular}
\end{table}

\footnotetext{``\emph{Yvyra pykue puku}'' ou ``raiz comprida da árvore''
  é uma metáfora que a cobra Jararaca utiliza para comparar a extensão
  do seu corpo com a da raiz das árvores.}


Upéicha Jarara guahu oiko yvypóra ramo.

\begin{quote}
Assim é o canto \emph{guahu} da Jararaca.
\end{quote}

\chapter{Ju'i guéῖ guahu / O canto da Perereca}

Ju'i guéῖ ymã ñande ypy reko ñeypyrũ ramo araka'e ko yvy ári, ju'i oheja
oñe'ẽngáry guahu gueko etépygua oñeha'ã hag̃ua yvy pýpe. Ju'i guéῖ niko
he'i oñe'ẽngáry rupi omombe'u yvyraku'a ku'ahápy oiko'akue oguenohẽ
oguahu rupi he'i:

\begin{quote}
Perereca\footnote{Os Kaiowa têm sua própria forma de classificação dos
  anfíbios. \emph{Ju'i gueῖ} é o termo utilizado para se referir às
  pererecas, que podem ser encontradas nas cores verde, amarela e cinza
  nos territórios kaiowa. Já \emph{ju'i mbui}, que é chamada no
  português dos Kaiowa de rã, é maior, é comestível e tem outro tipo de
  canto, assim como seu próprio \emph{guahu}. Existem outras, como por
  exemplo: \emph{ju'i}, uma rã comestível menor que \emph{ju'i mbui},
  comum na região do Ka'aguyrusu; \emph{ytu}, um sapo muito pequeno que
  vive na terra, em locais úmidos, mas frescos, e que pode ser ouvido a
  longas distâncias; \emph{ju'i kara}, uma perereca menor do que
  \emph{ju'ῖ gueῖ} e que se diferencia por meio de seu canto.} de cor
verde, chamada pelos Kaiowa de \emph{ju'i guéĩ}, no princípio da vida,
também fez questão de deixar o seu canto \emph{guahu} para os seres da
terra, conforme a sua voz e seu jeito de cantar. Perereca, para cantar,
sobe no tronco da árvore e fica bem posicionada. De lá, Perereca canta
sobre seu jeito de ser, assim:
\end{quote}

\begin{table}[]
\begin{tabular}{ll}
Guéῖ guéῖ                     & Guéῖ guéῖ                             \\
Ere Ju'i                      & Canta a Perereca                      \\
Guéῖ guéĩ                     & Guéῖ guéĩ                             \\
Ere Ju'i                      & Canta a Perereca                      \\
Yvyra ku'a peῖ ereñe'ẽko Ju'i & No tronco da árvore, canta a Perereca \\
                              &                                       \\
Guéῖ guéῖ                     & Guéῖ guéῖ                             \\
Ere Ju'i                      & Canta a Perereca                      \\
Guéῖ guéῖ                     & Guéῖ guéῖ                             \\
Ere Ju'i                      & Canta a Perereca                      \\
Yvyra ku'a peῖ ereñe'ẽko Ju'i & No tronco da árvore, canta a Perereca
\end{tabular}
\end{table}

Upéicha ojehe ju'i he'i. Ju'i niko jasareko ramo hese oñe'ẽ hag̃ua ojupi
voi yvyraku'ápy ha upérupi oñendu uka uka arã oky rag̃ua ikatueteháicha
ohenói amápe. Upéagui pyhare pukukue jeho guahuhápy oñeha'ã mante voi
ju'i guahu oheja iñe'ẽngaryháicha.

\begin{quote}
No momento em que a Perereca canta, onde ela está? Difícil perceber --
pode estar escondida atrás do tronco da árvore. Perereca canta na
véspera da chuva ou, por meio de seu canto, chama a chuva. Faz todo o
possível de chamar a chuva. Desde então, o \emph{guahu} da Perereca é
cantado em algumas noites, depois de rituais importantes.
\end{quote}

\chapter{Akutipáy guahu / O canto da Cutia}

Akuti reko ypy rehegua ñemombe'upy, ha'e niko ohekokuaa araka'e
rembypyrã, ha upérupi ombojehu ha ombojeroreta yvypýpe. Akuti ka'aguy
rupi oiko kuaa voi, ombogua kuaa monde hape rupi ojejapo ramo, hesa katu
yvyra guyrehe, ha upéagui ndoikereíry monde guýpe. Ñande ramõi vete
omombe'u va'ekue rupi ete akuti reko rehegua; ha'e osẽ oguata ha okaru
hag̃ua ka'aguy rupi ojehovasa rañe katuete opa mba'e va'i omboguakatu
raguã oguapegui, ha ojehovasa rire katu oguatáma ka'aguy rupi ha
oñangareko voi opamba'e rehe. Upéagui, monde ojejapo ramo akuti rape
rupi ka'aguy mbyte rupi, monde japoha oinupã nupã rañearã ñandy ta'ypy
ha uperire katu oñakã upi hag̃ua mondépe, upéa rire ndaje monde
ndaijayvúiry akuti oúramo hupi. Ha upéicha háguy ombovy joty ramo monde,
akuti pypore yvy reheve ombohasa arã monde guykoty, upéicha rire mante
ha'e oike monde guýpe. Upéagui akuti oguahu rupi oipapa ñembohory hápy
monde ombogua va'ekue rehegua, ha upéicha he'i:

\begin{quote}
No início da existência, Cutia recebeu orientações imprescindíveis para
sua vida; depois disso, ela surgiu e se multiplicou aqui na terra. Cutia
sabe como se desviar da situação perigosa no mato. Cutia possui
inteligência natural e sabe como desarmar armadilha quando é colocada em
seu caminho. Cutia, ao andar na mata, observa atentamente qualquer
madeira pendurada, por isso não é fácil de capturar no mundéu. Esta
história o nosso avô contava bem, sobre o modo de ser da Cutia: antes de
caminhar pela mata, Cutia faz os gestos do \emph{jehovasa}\footnote{\emph{Jehovasa}
  se refere um conjunto de gestos feitos com as mãos, por meio dos quais
  se podem obter diferentes efeitos, como afastar, atrair extirpar,
  proteger, entre outros, a depender do movimento feito pelas mãos.
  Trata-se de uma prática de difusão ampliada, conhecida pela maioria
  das pessoas, mas também empregada de forma especializada pelos
  \emph{ñanderu}, por exemplo nos processos de cura - quando pode ser
  acompanhado pelo sopro (\emph{peju}). Cutia, da mesma forma, faz
  \emph{jehovasa} não só para escapar do mundéu, mas para afastar todo e
  qualquer mal.}, desviando todos os males de seu caminho. Só depois de
fazer \emph{jehovasa}, Cutia sai caminhando no meio da mata, sempre
observando atentamente os arredores. Por esse motivo os Kaiowa, quando
fazem armadilha no trilheiro de Cutia, depois de pronto o mundéu, buscam
o galho com espinhos de uma planta chamada \emph{ñanduta'y}. Com ele,
golpeiam levemente a armadilha e só depois suspendem a madeira, deixando
ela pronta para capturar a Cutia. Segundo o conhecimento kaiowa, o
mundéu, depois de golpeado com o \emph{ñanduta'y}, deixa de fazer ruído,
e Cutia não percebe que é armadilha: só assim é que se captura a Cutia.
Por isso, Cutia, com seu canto \emph{guahu}, fala de sua esperteza,
debochando da aparência do mundéu. Ela canta assim:
\end{quote}


\begin{table}[]
\begin{tabular}{rl}
Gua ke ke         & Gua ke ke           \\
Gua ke ke         & Gua ke ke           \\
Monde mogua rire  & Escapando do mundéu \\
Amoguáko monde ju & Escapei do mundéu   \\
Amoguáko monde ju & Escapei do mundéu   \\
                  &                     \\
Gua ke ke         & Gua ke ke           \\
Gua ke ke         & Gua ke ke           \\
Monde mogua rire  & Escapando do mundéu \\
Amoguáko monde ju & Escapei do mundéu   \\
Amoguáko monde ju & Escapei do mundéu  
\end{tabular}
\end{table}

Upéicha Akuti guahu oipapa ojehegua.

\begin{quote}
Assim o \emph{guahu} da Cutia conta sobre ela.
\end{quote}

\chapter{Karumbe guahu / O canto da Tartaruga}

Karumbe ha jaguarete omenda ramo guare rehegua ñemombe'u. Ary ha yvy
oñepyrũ ha omboypy rire araka'e, jaguarete kuña ha karumbe kuimba'e
oñemboki ete araka'e ojoehe ha omenda, karumbe iguaná eterei ha oguata
mbegue ave, ijape katu ipekue hatã voi. Karumbe ndaikatúi oguata pya'e,
ipy ha ipo yvyrehe opoñy haguánte voi; upéagui ndaikatu mo'ãi voi ojupi
yvyvra rehe, ha jaguarete kuña katu ohayhu eterei ichupe, ndoikoséi
mombyry karumbégui. Peteĩ árype karumbe oho omangeko oguembireko ndive,
oñyvõ ñyvõ guyra yvategua ha iju'y opita ojepoko yvyra rakã rehe,
karumbe ojeupita mo'ã oguevivy iju'y rehe ha ndaikatúi ojeupi, ha
rembireko katu pya'epy ojeupi yvyra rehe omotyvõ oity karumbe hu'y
yvýpy. Upéa rire Karumbe opyta oñemotĩ eterei, ha ojesaity ojehe
ndaikatuiha oiko oguembireko oikoháicha. Upéagui karumbe oheja jaguarete
kuñápe, ha okañỹ mombyry oho guembirekógui. Jaguarete katu ohayhu
etereígui oho hapykuéri oheka karumbépe rekuaty kuaty rupi oñendu'uka
uka karumbépe, upe maramo karumbe oguahu rupi he'i:

\begin{quote}
Tartaruga e Onça, segundo contam os Kaiowa, se casaram. Depois da
criação do céu e da terra, a Onça fêmea e a Tartaruga macho ficaram
encantados um pelo outro, os dois se apaixonaram perdidamente e se
casaram. Tartaruga anda muito devagar e seu casco é muito duro. Ele não
pode caminhar rápido, a sua pata é arrastada pelo chão, caminha como se
estivesse se arrastando, este é seu jeito de ser, certamente não tem
condições de subir na árvore. A Onça amava o Tartaruga imensamente, não
podia ficar longe dele. Um dia Onça e Tartaruga foram caçar, levando
arco e flecha. Tartaruga, em certo momento, atirou em um pássaro, mas a
sua flecha ficou presa num galho de árvore. Tartaruga tentou subir no
tronco da árvore de cabeça para baixo e não conseguiu. A Onça percebeu
que seu marido Tartaruga não podia subir na árvore, então subiu
rapidamente no tronco da árvore e desprendeu a flecha do galho,
deixando-a cair no chão. Depois desse episódio, Tartaruga ficou muito
envergonhado, olhou seu corpo e percebeu que não podia andar como a sua
esposa. Depois disso, Tartaruga abandonou a sua esposa Onça e foi embora
bem longe. A Onça ficou bastante preocupada e seguiu Tartaruga até o
lugar onde vivia seu amado, gritando e gritando e chamando o nome de
Tartaruga. Em certo momento, Tartaruga ouviu os gritos, e então cantou:
\end{quote}


\begin{table}[]
\begin{tabular}{rl}
Eru che rembeta chévy ha'i          & Traga-me meu tembetá, minha mãe           \\
Eru che rembeta chévy ha'i          & Traga-me meu tembetá, minha mãe           \\
Eru che rembeta chévy ha'i  		& Traga-me meu tembetá, minha mãe \\
Eru che rembeta chévy ha'i 			& Traga-me meu tembetá, minha mãe   \\
                  &                     \\
Taha tahecha         & Vou lá ver           \\
Aipo kuña chehegui poindare         & Aquela mulher que eu abandonei           \\
Guetyma jukeri'y jepe  & Ajoelhada no tronco do Juqueri\footnotemark{} \\
Kuña ndoikuaái & Mulher que não quer entender   \\
                  &                     \\
Taha tahecha         & Vou lá ver           \\
Aipo kuña chehegui poindare         & Aquela mulher que eu abandonei           \\
Guetyma jukeri'y jepe  & Ajoelhada no tronco do Juqueri \\
Kuña ndoikuaái & Mulher que não quer entender   \\
\end{tabular}
\end{table}

\footnotetext{Nesse verso a Tartaruga fala da Onça, a mulher que insiste
  no homem que a abandonou. O Juqueri é uma árvore de tronco espinhoso,
  então o verso metaforiza os sofrimentos dessa mulher amargurada, que
  não quer se afastar desse homem.}

Upéicha ha'ekuéry ojotopa jevy y rembe'ýpy, Karumbe katu oguembirekógui
ojepoi ýpy ha oyguy, ha Jaguarete ojepoi ave hapykuéri ha haimete omano
y guýpy, Jaguarete nomanoséi Karumbe rehehápy opoi ete karumbégui.

\begin{quote}
Onça e Tartaruga se encontraram novamente na beira de um rio. Tartaruga,
de tanta alegria, se atirou na água e ficou submerso por alguns minutos.
A Onça também se atirou atrás de Tartaruga, mas quase se afogou na água.
Desde então, a Onça fêmea resolveu se separar para sempre de Tartaruga,
pois a Onça não quer morrer só por causa de Tartaruga.
\end{quote}

\chapter{Kapiyva guahu / O canto da Capivara}

Kapiyva ypy oipapa ijehegua omombe'u yrykóta rehe guare. Kapiyva oipapa
ogueko rehegua oguahu rupi ñande rembypy kuérype araka'e, ñemombe'u pyrã
voi yvypórape ojeroeta opa tetã guasu hárupi guarã. Kapiyva ogueko
rehegua tesa rupi guarã voi oipapa, mba'éichapa rekuaty ojekuaa va'erã
yrykóta rehe yugua rusu rusu oikoha raty rupi. Kapiyva niko ojehesaity
ha ovy'avéva yugua rusu kóta rehe voi, ha upéagui ha'e oipapa omboyvy
ku'i ku'i oñenoháty oikóvy ramogua. Kapiyva oikoha rupi niko yugua rusu
rembe'y ipotῖ voi, ha'e iñenõha yrykóta ha ikaruháty rupi ndahokýi
ñahãna ra'y, ha'e voi ho'u kapi'i kyre ogueta reheve, upéagui omopotῖ
oguekuaty. Kapiyva oikoeteve yrykóta rehe ha juvy'y rehe, ha ohendu
hapykuéri jeiko ramo, ohepy rag̃ua oguekove ha'e ho'a ave ýpy ha oyta
yguy rupi mombyryhápy osẽ okañy rag̃ua juvy'y rehe. Ha yugua guasu
ojahuháty py katu omohypytĩarã y, ojahu ha ojavyky etereígui ombohyjúi
hyjúi y syry. Upéagui kapiyva oguahu rupi oipapa oguekoha ha oguekuaty
rehegua, ha upéicha he'i:

\begin{quote}
Capivara, conforme a narrativa kaiowa, possui sua história, e ele mesmo
conta sobre o seu ambiente de vida. Capivara narra por meio do canto
\emph{guahu} para as divindades narrativas para repassar de geração a
geração, em cada território. Capivara conta sobre seu ambiente de vida,
algo que é visível nas beiras de rio ou lagos por onde anda. Capivara
gosta de viver na beira da água ou lago, onde é sempre limpo e existe
muita lama, ou poeira, onde Capivara se deita para descansar. Capivara
gosta de comer capim novo e gosta de viver na beira da água ou no brejo,
mas, quando percebe o perigo ou é atacado pelo caçador, corre e cai na
água, ficando por alguns minutos debaixo d'água. Às vezes, sai do outro
lado do córrego e vai embora. No lugar onde vive, a água sempre fica
suja, pois a capivara a todo o momento cai n'água, ou às vezes deixa a
água espumando. Por isso, Capivara, por meio de seu canto \emph{guahu},
narra como é seu ambiente e como brinca com seus companheiros, cantando
assim:
\end{quote}

\begin{table}[]
\begin{tabular}{rl}
Yvy ku'i guasu rupi che guahapa          & No imenso pó da terra eu brinco           \\
Yryjúipy ajoapi          & Espumas d'água eu lanço           \\
Yryjúipy ajoapi  		& Espumas d'água eu lanço \\
                  &                     \\
Yvy ku'i guasu rupi che guahapa          & No imenso pó da terra eu brinco           \\
Yryjúipy ajoapi          & Espumas d'água eu lanço           \\
Yryjúipy ajoapi  		& Espumas d'água eu lanço \\
\end{tabular}
\end{table}

\chapter{Guairaka guahu / O canto da Lontra}

Guairaka oguahu rupi oipapa ojehegua omoherakuã yvypórape oguekóypy.
Guairaka rembi'u niko pira añónte voi, pira rehe ha'e ojeporeka kuaa y
rupi okaru hag̃ua; upe hekorã hembypy araka'e omboypy ichupe. Upéagui
ikaruhaty hague rupi y rembe rehe hetave hetave pira pekue omosãrambi.
Upéichagui guairaka ojehegua omombe'u ha oñemomba'e guasu voi ha'e
añónte imba'e juka kuaaha, ha upéagui rese mbavéva ndaija'éi. Guairaka
ojehe he'i ha'eñónte hetaha hembireko oikóva hembijukakue rehe,
upéichagui ha'e oñemboete voi, ha iguahu rupi oñemombe'u ha oipapa
ogueko ipira jukaha guembirekópe. Upéicha hekógui ave he'i ojehe
guairaka omojesaity ha ombotavyha ijehe guapicha rembireko, upéa ha'e
oguerohehe hembireko retaha. Guairaka reko ete ypy niko ha'e hãi ojekuaa
ichupe hu'y ramo ha upéapy oñyvõ pirápe ojuka ha ho'u hag̃ua, ha upéa
va'e gueko guairaka omomba'e tee voi ha'e añónte ikatupyry ha ojuka
kuaaha pirápe. Upéagui guairaka omoñesyrũ oguahu ijehegua meme omombe'u.

\begin{quote}
Por meio de seu canto \emph{guahu}, Lontra conta seu comportamento e
como vive com outro animal. Como Lontra caça os peixes para comer, pois
gosta de comer peixe de qualidade, seu dente é uma flecha letal para
matar peixe - as divindades deram uma flecha para seu uso de
sobrevivência. Por esse motivo, por onde Lontra vive e come só os peixes
deixa espalhada a escama de peixe na beira do córrego ou rio só para
demonstrar que come bem e sabe capturar os peixes. {[}Por{]} essas
provocações os outros animais que também comem peixe não gostam de
Lontra. Lontra também gosta de fazer outras provocações: por sua
natureza, Lontra gosta de ter várias fêmeas como suas companheiras,
vivendo em grupo. Cada lontra possui o seu território para viver com
várias lontras fêmeas. O macho caça mais para as fêmeas comerem. Por
isso, Lontra provoca os outros bichos, dizendo que só ele tem várias
esposas e outros não têm. Lontra também às vezes rouba fêmeas de outro
grupo e assim fica feliz vivendo no seu território. A flecha de Lontra é
seu dente, e ele se considera o melhor caçador de peixes de todos os
tempos. Este é o seu jeito, que orgulhosamente narra por meio de seu
canto \emph{guahu}.
\end{quote}

\begin{table}[]
\begin{tabular}{rl}
Che katu ava mba'e juka          & Eu sou um bom caçador           \\
Che katu ava mba'e juka          & Eu sou um bom caçador           \\
                  &                     \\
Che rehe meme niko  		& Todos falam de mim \\
``Pende rembireko reta''          & ``Suas esposas são muitas''           \\
Che rehe meme niko          & Todos falam de mim           \\
``Pende rembireko reta''  		& ``Suas esposas são muitas'' \\
                  &                     \\
Che katu ava mba'e juka          & Eu sou um bom caçador           \\
Che katu ava mba'e juka          & Eu sou um bom caçador           \\                  
                  &                     \\
Che rehe meme niko  		& Todos falam de mim \\
``Pende rembireko reta''          & ``Suas esposas são muitas''           \\
Che rehe meme niko  		& Todos falam de mim \\
``Pende rembireko reta''          & ``Suas esposas são muitas''           \\
Che rehe meme niko  		& Todos falam de mim \\
``Pende rembireko reta''          & ``Suas esposas são muitas''           \\
                  &                     \\
Che katu ava mba'e juka          & Eu sou um bom caçador           \\
Che katu ava mba'e juka          & Eu sou um bom caçador           \\
\end{tabular}
\end{table}


\chapter{Pykasu guahu / O canto da Pomba}

Pykasu oipapa iguahu rupi iguapy raty rehegua, yvyra rakã hyveraha upéa
va'e omombe'u. Añetéma, pykasu ary pukukuépy mombyry mbyry oveve ojeheka
okaru rag̃ua, ha ojevykuévy katu pykasu oñemohenda arã yvyra rakã rehe,
ha upépy ovy'a ramo oñe'ẽ voi ikatuhápe peve, ha ndovy'ái ramo katu
oñe'ẽ ñembyasy asy oñombohováivy ka'aguy ijyvateha rehe. Yvyra rakã
pykasúpe ojekuaa hyvera vera porã ipotῖgui iguapy hatýgui. Upéa ha'e
oipapa oguahu rupi ha omomba'e guasu ave guekokuéry. Upéagui iguahu
he'i:

\begin{quote}
Pomba, por meio de seu canto \emph{guahu}, conta sobre seu modo de vida,
seu espaço de circulação, e como vive no galho das árvores. Verdade,
pois, quando observamos a Pomba, ela viaja e voa, e vai longe para
procurar alimento, e no entardecer retorna ao seu espaço e passa alguns
momentos cantando para se alegrar, ou às vezes canta de forma triste.
Nos galhos das árvores, muitas vezes a Pomba canta, alegrando outros
pássaros. Ela narra como, por meio do seu canto, o galho das árvores se
transforma e brilha como um relâmpago. Assim, por meio de seu canto, ela
conta sua história e exalta seu modo de viver. Por isso seu \emph{guahu}
diz:
\end{quote}


\begin{table}[]
\begin{tabular}{rl}
Overa vera porã          & Relampeia belamente           \\
Overa vera porã          & Relampeia belamente           \\
Che guapyha  		& Meu assento \\
Che guapyha          & Meu assento           \\
                  &                     \\
Overa vera porã          & Relampeia belamente           \\
Overa vera porã  		& Relampeia belamente \\
Che gueraha ramo          & Me carregando           \\
Che gueraha ramo          & Me carregando           \\                  
                  &                     \\
Pykasu he'i guyra  		& ``É a pomba'', dizem os pássaros \\
Pykasu he'i guyra          & ``É a pomba'', dizem os pássaros           \\
\end{tabular}
\end{table}




Ñande yvypóra iporã jaikuaa guyra pykasu rehegua ñamombe'u rag̃ua ñande
re'yikuérype.

\begin{quote}
Nós que somos aqui da terra precisamos aprender sobre a Pomba para
recontar às novas gerações.
\end{quote}

\chapter{Kaguare guahu / O canto do Tamanduá-bandeira}

Kaguare\footnote{Kaguaré, o tamanduá-bandeira, também é chamado pelos
  Kaiowa de \emph{jarutare} em algumas regiões. Já o Tamanduá-mirim é
  chamado por eles de Kaguaré-mirim.} ypy ojehu hese va'ekue omombe'u
oguahu rupi. Kaguare niko oguahu rupi ojehegua oipapa oyvy monã oiko
ramo guare, kaguare oikoeteve takuru ha kupi'i reta oimehã rupi, ha
upéare ojeporaka ha okaru hag̃ua. Kaguare ndoguerekói hãi, upéagui ha'e
oiporu opoapẽ omohãi kupi'i ho'u rag̃ua, ha ipochi ramo katu ipoapẽ
añónte voi ave oiporu ojekoepy rag̃ua. Kaguare oguata ramo ka'aguy rupi
opa mba'ety rupi voi oguata ohekávy kupi'i ho'u rag̃ua. Upéa rekoha
omombe'u oguahu rupi yvypórape ombovy'a rag̃ua reko ñemboasýva ha
ivy'are'ỹ va'épe ave. Upéagui kaguare guahu re'i:

\begin{quote}
Tamanduá-bandeira, no início da sua existência, passou por várias
situações complicadas. Esses fatos que aconteceram na sua vida são o que
o tamanduá conta em seu canto \emph{guahu}. Tamanduá-bandeira fala de
suas andanças de um lugar para outro à procura de cupins. Tamanduá não
tem dentes. A sua arma de defesa é sua unha comprida, que usa para
desmanchar os cupinzeiros. Tamanduá-bandeira, quando fica irritado ou se
sente em perigo, usa as unhas para defender a sua vida. Quando anda na
mata, ele pode passar no meio do caraguatá e outras plantas espinhosas à
procura dos cupins. Suas andanças ele conta, por meio do \emph{guahu},
para os seres da terra e para alegrar aos que estão tristes. Por esse
motivo, o \emph{guahu} do Tamanduá-bandeira é assim:
\end{quote}

\begin{table}[]
\begin{tabular}{rl}
Oñuasa\footnotemark{} asa porã          & Passeando pelo campo belo           \\
Oñuasa asa porã          & Passeando pelo campo belo           \\
Oñuasa asa porã  		& Passeando pelo campo belo \\
Oñuasa asa porã          & Passeando pelo campo belo           \\
Kaguare          & Tamanduá           \\
Kaguare          & Tamanduá           \\                  
\end{tabular}
\end{table}

\footnotetext{O termo \emph{oñuasa}, encontrado nos cantos e no
  vocabulário dos mais velhos, é usado para indicar a circulação do
  tamanduá pelo campo (\emph{ñu}). Aqui optamos por traduzir como um
  passeio do tamanduá.}


``Upéicha che omboyvy porã che \emph{guahu}'', he'i Kaguare.

\begin{quote}
``Isso é o que quero deixar para os seres da terra'', diz Tamanduá.
\end{quote}

\chapter{Inambuju guahu / O canto do Jaó}

Inambuju je'e guyra hohõpe, ha ojehenói ave ichupe kokoépy. Guyra Hohõo
oiko ka'aguy jyty yvy ete rehe voi oguata, ha'e rekuaty ka'aguy jyty
potῖ rupi oiko ha okaru opa mba'e ñahãna poty ha yva aju ra'yῖngue rehe,
ha oveve ramo katu ndoho pukúi oguejy jevý ma yvýrehe. Upéicha Hohõo
reko, ha'e oguatahárupi oñenduuka uka oguahu rupi oikóvy. Ymã ete Hohõo
reko ypykuéry he'i araka'e amondóta inambuju ijayvu yvu hag̃ua yvyra jyty
rehe voikue ha ka'arukue upéicha araka'e omboypy ichupe rekorã. Upéagui
Hohõo oguahu porã porã ka'aguy jyty rehe ombovy'a rag̃ua ary ka'aguy
omoypy ypy ichupe haguéicha oiko. Upéa Hohõo oipapa oguahu rupi he'i:

\begin{quote}
Esse pássaro é o que os Kaiowa chamam de \emph{Hohoõ} e também de
\emph{Kokoe}. O Inambuju vive na mata, especificamente mata baixa e
limpa, para procurar a sua comida. Ele se alimenta de alguns insetos e
sementes de plantas nativas. Voa no momento preciso, mas não muito
longe, e vive revirando as folhas secas das árvores para encontrar seus
alimentos. \emph{Inambuju} é um pássaro que vivia muito próximo das
divindades, mas depois elas mandaram que ele vivesse no mato e cantasse
a cada amanhecer e entardecer. Desde então, \emph{Hohõo} canta bem suave
a cada amanhecer e entardecer no mato. O canto de Inambuju não é apenas
um canto, é um \emph{guahu} que narra o momento triste de seu
afastamento, mas também alegra os outros pássaros e a floresta. Assim
Jaó conta sua história:
\end{quote}


\begin{table}[]
\begin{tabular}{rl}
Amondo Inambuju          & Mandei o Jaó           \\
Iñe'ẽ kamavarῖ\footnotemark{}          & Com seu canto singular           \\
Amondo Inambuju  		& Mandei o Jaó \\
Iñe'ẽ kamavarῖ          & Com seu canto singular           \\
                  &                     \\
Oñe'ẽ Inambuju\footnotemark{}          & Canta o Jaó           \\
Oñe'ẽ Inambuju          & Canta o Jaó           \\
Oñe'ẽ Inambuju  		& Canta o Jaó \\
Oñe'ẽ Inambuju          & Canta o Jaó           \\
                  &                     \\
Amondo Inambuju          & Mandei o Jaó           \\
Iñe'ẽ kamavarῖ          & Com seu canto singular           \\
Amondo inambu ju  		& Mandei o Jaó \\
Iñe'ẽ kamavarῖ          & Com seu canto singular           \\
                  &                     \\
Yvyra jy yvyra jyty katu pei          & Entre as árvores e as folhas caídas           \\
Erene'ẽko          & Ele está cantando           \\
Yvyra jyty katu pei  		& Entre as árvores e as folhas caídas \\
Erene'ẽko          & Ele está cantando           \\
                  &                     \\
Amondo Inambuju          & Mandei o Jaó           \\
Iñe'ẽ kamavarῖ          & Com seu canto singular           \\
Amondo Inambuju  		& Mandei o Jaó \\
Iñe'ẽ kamavarῖ          & Com seu canto singular           \\
\end{tabular}
\end{table}



\footnotetext{O termo \emph{kamavarῖ} indica a voz específica, singular do Jaó.}

\footnotetext{A partícula \emph{ju} no termo \emph{Inambu ju} se refere
  ao \emph{jeguaka} (diadema) do Jaó. Novamente, não se trata de uma
  cor, mas do aspecto resplandecente do Inambu.}




Upéicha Hohõo guahu ojepapa.

\begin{quote}
Assim canta o Jaó, contando do seu passado.
\end{quote}

\chapter{Guyra pipiu guahu / O canto do Piu-piu}

Guyra Pipiu ojehekombo'e araka'e oikoha rupi imba'e poroa'o rag̃ua voi.
Ha'e mba'e ohecha ramo yvyrehe ha yvyra rakãre oja'o ja'óma voi
ikatueteháicha ichupe. Guyra Pipiu oiko eteve ka'aguy ikarape ha yvyra
isaikãty rupi voínte, ha ka'aguy ijyvate ete háre oho ramo katu oheka
hag̃uánte voi oguembi'urã ñahãna poty ha yvyra poty ho'u rag̃ua. Oguataha
rupi guyra Pipiu ohecha ramo yvýre oguatáva oja'óma voi, upéagui iguahu
he'i:

\begin{quote}
O pássaro Piu-piu\footnote{Trata-se do \emph{Myrmorchilus strigilatus},
  também conhecido como ``piu-piu'' no Brasil. É uma ave passeriforme,
  pequena, que se alimenta de insetos, e vive em dois habitats secos da
  América do Sul: o Chaco do Paraguai, Uruguai e Bolívia, e a Caatinga
  do Nordeste do Brasil.}, em seu princípio, foi ensinado a acuar por
meio de seu canto. O Piu-piu vive na floresta baixa e procura seus
alimentos na mata fechada. Piu-piu é esperto e, quando observa algo
anormal, ou ouve um ruído estranho nos arredores, repetidamente avisa
outros pássaros para que não se aproximem daquele lugar. Por onde voa
ou, quando procura seus alimentos, o Piu-piu sempre busca alguma flor
das árvores para se alimentar. Ele protege os outros pássaros ao
alertá-los do perigo, por isso seu \emph{guahu} é assim:
\end{quote}


\begin{table}[]
\begin{tabular}{rl}
Che'ao che'ao          & Me azucrina, me azucrina           \\
Guyra pipiu, guyra pipiu          & Pássaro Piu-piu, pássaro Piu-piu           \\
Guyra poroa'o  		& Pássaro que me azucrina \\
Guyra poroa'o          & Pássaro que me azucrina           \\
Guyra pipíko chea'o          & Pássaro Piu-piu me azucrina           \\
Chea'o          & Me azucrina           \\
\end{tabular}
\end{table}

Upéicha iguahu te'ýi oikoha rupi oipapa chupe ihekóre meme ombopapa.

\begin{quote}
Assim é o \emph{guahu} do Piu-piu que os Kaiowa cantam.
\end{quote}

\chapter{Mbarakajaju guahu / O canto do Gato Dourado}

Mbarakajaju oipapa hesegua yvy rendy rupi oiko ramo guare ha omombe'u
yvyporape iguejy guejy ratygue. Mbarakajaju oiko tarova ramo guare yvy
rendy rekoetépy ha'e oguejy pya'épy araka'e ysypo rupi ochuka rag̃ua
ogueko añetehapygua, upéicha Mbarakaja ypy ochuka oguekoha araka'e, ha
upéa hekokue oipapa chupe ñande ypy, ha'e i\emph{guahu}rã he'i:

\begin{quote}
O Gato Dourado conta sobre quando ele vivia em seu patamar sagrado e
revela para as pessoas da Terra por onde ele descia até aqui. Em um
momento difícil de sua vida, o Gato Dourado desce por um cipó para
mostrar sua plenitude por alguns instantes aqui na Terra. Assim, ele
mostrava, e assim reconta, em seu \emph{guahu}, esse passado:
\end{quote}



\begin{table}[]
\begin{tabular}{rl}
Ysypo guejy rupi ereguejy          & Por meio do cipó desce           \\
Mbarakajaju\footnotemark{}          & O gato dourado           \\
Mbarakajaju 		& O gato dourado \\
                  &                     \\
Ysypo guejy rupi ereguejy          & Por meio do cipó desce           \\
Mbarakajaju          & O gato dourado           \\
Mbarakajaju          & O gato dourado           \\
\end{tabular}
\end{table}


\footnotetext{Aqui optamos por traduzir a partícula \emph{ju} como
  dourado, como uma forma de expressar o aspecto resplandecente deste
  gato específico, \emph{Mbarakajaju} é o guardião do \emph{chiru},
  bastão utilizado pelos xamãs, \emph{ñanderu}.}




Upéicha Mbarakajaju guahu oñeha'ã ko yvypýpe.

\begin{quote}
Assim o \emph{guahu} do Gato Dourado permanece aqui na terra.
\end{quote}

\chapter{Piririju guahu / O canto do Anu-branco}

Guyra Piririju upéicha oñehenói guyra piririguápe. Ha'e rekuaty yvyra
aty ñuharegua, ha'e oiko eteve yvyra aty ñu rehe ho'u rag̃ua yvyra raso,
piririju ndohói ka'aguy rehe ha ndovevéi ave yvate eterei, ha'e oiko
eteve yvyra karape rakã rehe oguetakue reheve ha jai avoraity oĩmehã
rupi ave. Piririju oguãhẽ jave áry ka'a roky oñe'ẽ vy'a opa rupi rei
ñuty rehe, ha oñemboasy ramo katu oikoha rupi oñe'ẽ po'i puku puku
oikóvy; upéicha ramo piririju ojeupi'a eru áry pukukue rupi. Upéagui
iguahu chupe ojeheja akue he'i ohóvy:

\begin{quote}
Piririju: assim é chamado o Anu-branco. Seu \emph{habitat} são os campos
ou capoeiras onde há poucas árvores, pois o Anu-branco procura insetos
na vegetação baixa para se alimentar. No período de sol quente e no
tempo em que brota a vegetação, especialmente no mês de agosto,
Anu-branco canta alegre na sombra das árvores. Às vezes canta meio
triste -- nesse período, está pondo ovos no ninho. Por isso, o
\emph{guahu} do Anu-branco é assim:
\end{quote}


\begin{table}[]
\begin{tabular}{rl}
Ano piririju          & Anu piririju\footnotemark{}           \\
Ereñe'ẽko guyra          & Fala o pássaro           \\
Ano piririju 		& Anu piririju \\
Ereñe'ẽko guyra          & Fala o pássaro           \\
                  &                     \\
Jai avorái atýra rehe 		& No emaranhado de cipó\footnotemark{} \\
Ereñe'ẽko guyra          & Fala o pássaro           \\
\end{tabular}
\end{table}

\footnotetext{Os Kaiowa aproximam o Piririju ou Piririgua (\emph{Guira
  guira,} anu-branco) dos pássaros Anu por não voarem muito alto nem
  percorrerem grandes distâncias. Por isso, aqui, ele também é chamado
  de Anu. Novamente, a partícula ju no nome do pássaro faz referência ao
  seu diadema (\emph{jeguaka}) resplandecente.}


\footnotetext{O verso faz referência à árvore coberta de cipós em que o
  Anu-branco costuma descansar e dormir durante a noite.}


Upéicha Piririju guahu oiko ko yvypýpe omomba'e guasu guasu ichupe.

\begin{quote}
Assim foi que Anu-branco, um pássaro importante na Terra, teve a origem
de seu \emph{guahu}.
\end{quote}

\chapter{Suru'a guahu / O canto de Surucuá}

Koa ñemombe'u pyre ojehu va'ekue Jari rehe, Ñanderu kuéra oiko joty
oñondivepa ko yvy pype ojehu va'ekue Jari rehe. Peteỹ ára Ñamõi oho
ojeheka ka'aguy rehe ha ojuka peteĩ tatu. Upéi Ñamõi ojevy oguahẽ hoypy,
Jari ojapo hu'i tatu tyrã. Ha ojapo jave, Ñamõi ha hemiarirõ ho'upáma
ichugui tatu ro'o. Jari opyta ñemoyrõ hemiarirõ ha iména ndive araka'e,
ha ka'aguy rehe oho oiko. Upéagui rire, Jari oñemboguyra ha guyra reko
rami oiko ka'aguy rupi, oveve veve oikóvy yvyra rakã rehe. Upéa heko
rehe, Ñanderu ombopyta ete jari guyra suru'a ypy ramo, ha ko'ãnga oiko
ka'aguy rusu rehe, oñendu uka uka oñe'ẽ ñembyasy. Ko'anga ka'aguy rusu
rehe oñe'ẽ ramo Suru'a itetávy he'i arã ñamõi: ``Aháta agueru Jari hu'i
ichugui ha'u hag̃ua'', omoñe'ẽnguévy chupe ha oho ramo oñehendu uka
hárupi Suru'a noñe'ẽvéima oñemoyrõgui. Jari ojeheko rerova ete suru'a
ypy oñemoyrõgui re'i, upéagui oñe'ẽ ñembyasy asy oikóvy ka'aguy rovy
rupi.

\begin{quote}
Esta história conta um fato que aconteceu com \emph{Jari}, Vovó, no
tempo em que as divindades ainda moravam todas juntas aqui na terra. Um
dia, \emph{Ñamoῖ}, Vovô, foi caçar no mato e matou um tatu. Depois Vovô
voltou para sua casa e a Vovó estava preparando \emph{hu'í,} farofa de
milho, para comer com carne de tatu. Mas ainda no processo de preparo,
Vovô e seu neto comeram toda a carne de tatu. Vovó ficou magoada com seu
neto e com seu marido, e foi viver no mato. Depois disso, Vovó se
transformou em pássaro e passou a viver como os pássaros vivem na mata,
voando e voando no galho das árvores. Com essa vivência, Ñanderu
transformou a Vovó para sempre no princípio do pássaro Surucuá, cantando
para os outros ouvirem seu canto melancólico. Hoje, na grande mata,
quando canta Surucuá, o Vovô fala assim para provocá-la: ``Vou lá roubar
\emph{hu'i} de Jari'', querendo alegrar Surucuá, mas, quando vai na
direção dela, Surucuá para de cantar, fica triste e já não canta. Vovó
vive eternamente triste como pássaro Surucuá, por isso canta suavemente
pela grande verde mata.
\end{quote}

\begin{table}[]
\begin{tabular}{rl}
Ka'aguy ra'e, ereñe'ẽko guyra          & Naquela floresta, canta o pássaro           \\
Ka'aguy guasu ra'e, ereñe'ẽko guyra          & Naquela floresta, canta o pássaro           \\
                  &                     \\
Suru'a guasu, ereñe'ẽko guyra          & Grande Surucuá, canta o pássaro           \\
Suru'a guasu, ereñe'ẽko guyra          & Grande Surucuá, canta o pássaro           \\
                  &                     \\
Ka'aguy mbyte ra'e, ereñe'ẽko guyra  & No meio daquela floresta, canta o pássaro           \\
Ka'aguy mbyte guasu ra'e, ereñe'ẽko guyra & No meio daquela grande floresta, canta o pássaro           \\
                  &                     \\
Yvyra mbyte ra'e, ereguapy reĩ guyra   & No alto daquela árvore, senta o pássaro           \\
Yvyra mbyte ra'e, ereguapy reῖ guyra   & No alto daquela árvore, senta o pássaro           \\
                  &                     \\
Suru'a guasu, ereñe'ẽko guyra          & Grande Surucuá, canta o pássaro           \\
Suru'a guasu, ereñe'ẽko guyra          & Grande Surucuá, canta o pássaro           \\
\end{tabular}
\end{table}

\chapter{Karaja guahu / O canto do Bugio}

Karaja oiko eteve ka'aguy rusu rupi voi yvyra rakã rehe, ha'e oky rag̃ua
ko'ẽju rupi oambakuái arã ikatu ete hápe peve oñe'ẽ hatã arã voi, ha
mombyrýgui oñehandu uka, upéicha ramo ha'e oikuaáma okytaha. Karaja
ijecháy naiporãi voi, upéagui iguahu rupi oñemombe'u pono hag̃ua ijatuapẽ
rehe ojepuka, ha'e yvýrupi oguata ohóvy ramo ivai voi ijatukupe; upéare
Karaja he'i oguahu rupi:

\begin{quote}
Bugio vive na mata grande e nos galhos altos das árvores. Na véspera da
chuva, o Bugio solta uivos repetidamente e, sempre no amanhecer, esse
som é ouvido à longa distância. Quando isso acontece, a chuva está
próxima para molhar a terra. Bugio sabe que tem aparência feia e, por
meio do canto \emph{guahu}, pede respeito a todos os seres da terra,
para não darem risada do seu jeito de caminhar e muito menos comentar
sobre sua corcunda. Quando o Bugio caminha, ninguém gosta do seu andar,
por isso em seu \emph{guahu} ele canta assim:
\end{quote}



\begin{table}[]
\begin{tabular}{rl}
Iñakuã ndai ndai poku ava\footnotemark{}  & Ele anda mais e mais rápido           \\
Iñakuã ndai ndai poku ava          & Ele anda mais e mais rápido           \\
                  &                     \\
Yvyra roty recháiko& Ele avista o perobal\footnotemark{}           \\
Yvyra roty recháiko& Ele avista o perobal           \\
Yvyra roty recháiko& Ele avista o perobal           \\
Yvyra roty recháiko& Ele avista o perobal           \\
                  &                     \\
Iñakuã ndai ndai poku ava   & Ele anda mais e mais rápido           \\
Iñakuã ndai ndai poku ava   & Ele anda mais e mais rápido           \\
                  &                     \\
Kokoty che váy ramo          & Naquela direção estou andando           \\
Kokoty che váy ramo          & Naquela direção estou andando           \\
Kokoty che váy ramo          & Naquela direção estou andando           \\
Kokoty che váy ramo          & Naquela direção estou andando           \\
                  &                     \\
Rejojái jojái teῖ che atuapẽ          & Não ria da minha corcunda           \\
Rejojái jojái teῖ che atuapẽ          & Não ria da minha corcunda           \\
                  &                     \\
Kokoty che váy ramo          & Naquela direção estou andando           \\
Kokoty che váy ramo          & Naquela direção estou andando           \\
                  &                     \\
Yvyra roty recháy ko          & Ele avista o perobal           \\
Yvyra roty recháy ko          & Ele avista o perobal           \\
                  &                     \\
Rejojái jojái teῖ che atuapẽ          & Não ria da minha corcunda           \\
Rejojái jojái teῖ che atuapẽ          & Não ria da minha corcunda           \\
Rejojái jojái teῖ che atuapẽ          & Não ria da minha corcunda           \\
Rejojái jojái teῖ che atuapẽ          & Não ria da minha corcunda           \\	
                  &                     \\
Iñakuã ndai ndai poku ava          & Ele anda mais e mais rápido           \\
Iñakuã ndai ndai poku ava          & Ele anda mais e mais rápido           \\
Yvyra roty recháiko          & Ele avista o perobal           \\
Yvyra roty rechái rire          & Ele avista o perobal           \\
                  &                     \\
Kokoty che váy ramo          & Naquela direção estou andando           \\
Kokoty che váy ramo          & Naquela direção estou andando           \\
                  &                     \\
Rejojái jojái teῖ che atuapẽ          & Não ria da minha corcunda           \\
Rejojái jojái teῖ che atuapẽ          & Não ria da minha corcunda           \\
\end{tabular}
\end{table}



\footnotetext{O termo \emph{ava} nesse verso indica que o bugio se vê
  como gente, \emph{ava}.}

\footnotetext{Os perobais eram grandes concentrações de árvores de
  peroba, uma mata grande e densa, típica da região sul de Mato Grosso
  do Sul. Com o avanço das frentes agropastoris ao longo do século XX e
  a expulsão dos Kaiowa de seus territórios, os perobais foram
  devastados, e sua madeira foi comercializada.}

Upéicha Karaja ojehe he'i. Karaja niko oĩme ára oyvy asa voi, oho hag̃ua
ka'guy mboypýri itarova rova voi; upéa va'e oguahu rupi karaja omombe'u.

\begin{quote}
Assim o Bugio fala sobre si. Para se esconder do outro lado do mato,
Bugio vai correndo, parecendo louco; assim o Bugio conta, por meio do
seu \emph{guahu}.
\end{quote}

\chapter{Ñandu guahu / O canto da Ema}

Ñandu omombe'u oguahu rupi ojehegua oñehenóiha takuru pytãpy, upéa va'e
ñandu ypy ojohu vai. Ñandu niko hekuaty ñu ha kokuere voi, ha áry
pukukue katu ha'e oheka voi opa mba'e yvyra raso ho'u rag̃ua oiko kuévy.
Ñandu okaru jave ñu ipotĩháre ha kokuere rehe ojehecha takuru kuru rami
mombyrýgui, upéagui je'e chupe takuru pytã jechaha rami nde jecháy he'i,
ombotavypa gue'ýi ambuekuérype. Upéagui iguahu he'i:

\begin{quote}
Ema narra por meio do \emph{guahu} um fato que aconteceu com ela no
princípio dos tempos. Ela foi chamada de cupinzeiro vermelho pelos
outros bichos e não gostou disso. A Ema vive geralmente nos campos ou
plantações. Durante o dia, ela procura os insetos para se alimentar. Às
vezes, procura comida no lugar onde existe muito capim e, neste momento,
de longe, se parece com um cupinzeiro. Por isso o \emph{guahu} da Ema
começa assim:
\end{quote}

\begin{verse}
Chembotavy kature\\
Chembotavy katurere

Takuru pytã jechaha he'i chévy ra'e

Takuru pytã jechaha he'i chévy ra'e
\end{verse}

\begin{verse}
Zombaram mesmo de mim\\
Zombaram mesmo de mim

Aquela que parece um cupinzeiro vermelho, assim me chamavam

Aquela que parece um cupinzeiro vermelho, assim me chamavam
\end{verse}

Upéicha Ñandu guahu oñeha'ã ko yvýpýpe.

\begin{quote}
Assim é cantado o \emph{guahu} de Ema.
\end{quote}

\chapter{Taguato guahu / O canto do Gavião}

Taguato ypy yma ete oiko araka'e yvy pype iñakã ratã ogoajáry rehe, oiko
hárupi mombyry mbyry voi osapukái po'i po'i ogoajárype ohenói oho rag̃ua
hendápy oiko vy'a sapy'a etemi oñondive rag̃ua; upéicha ha'e oiko araka'e
áry piraguai ha áry ryakuã oñepyrũ jave tekoha tetã guasu rupi ko yvy
pype. Hoajáry katu mombyry kuépy ojapysaka porã porã voi oikóvy
isapukáire ikatueeteháicha ohendu ramo oho rag̃ua ha'épy. Upéagui ñande
ypy kuéry ñamói ha jari ohendu ramo taguato oñe'ẽ puku puku áry pukukue
yvyra rakã rehe oikuaáma ogoajárype ohenõiha. Upéagui guahu ichupegua
he'i:

\begin{quote}
No início da existência, Gavião havia se apaixonado por sua cunhada. Ele
saía de sua casa e, de longe, ficava gritando para sua cunhada
repetidamente, chamando-a para ir até ele, para namorar escondido; assim
eles viviam no período de agosto a outubro. Ao ouvir o grito, a cunhada
procurava um jeito de sair da casa para se encontrar com Gavião. Às
vezes saía disfarçadamente e ia ao encontro dele. Por isso os Kaiowa, ao
ouvir o canto de Gavião, já sabem que ele está chamando sua cunhada. Por
isso, o \emph{guahu} de Gavião é assim:
\end{quote}

\begin{verse}
Yvyra apy ra'e, ereñe'ẽko guyra\\
Yvyra apy ra'e, ereñe'ẽko guyra\\
Yvyra apy ra'e, ereñe'ẽko guyra

Osapukái ogoajárype guyra\\
Osapukái ogoajárype guyra\\
Osapukái ogoajárype guyra\\
Osapukái ogoajárype guyra

Yvyra puku aja ra'e, ereñe'ẽko guyra\\
Yvyra puku aja ra'e, ereñe'ẽko guyra

Ogoajárype guyra\\
Osapukai ogoajárype guyra\\
Ogoajárype guyra\\
Osapukai ogoajárype guyra
\end{verse}

\begin{verse}
No alto daquela árvore, canta o pássaro\\
No alto daquela árvore, canta o pássaro\\
No alto daquela árvore, canta o pássaro

Gritava o pássaro para a sua cunhada\\
Gritava o pássaro para a sua cunhada\\
Gritava o pássaro para a sua cunhada\\
Gritava o pássaro para a sua cunhada

Enquanto aquela árvore comprida estiver de pé, canta o pássaro\\
Enquanto aquela árvore comprida estiver de pé, canta o pássaro

Para sua cunhada, pássaro\\
Gritava o pássaro para a sua cunhada\\
Para sua cunhada, pássaro\\
Gritava o pássaro para a sua cunhada
\end{verse}

Upéicha guahu he'i chupe.

\begin{quote}
Assim ele canta o seu \emph{guahu}.
\end{quote}

\chapter{Mainomby guahu / O canto do Beija-flor}

Ñande ypy kuérype araka'e guyra Mainomby ojekuaa ñanderu kuéry ñe'ẽ puku
va'ýra voi, ha'e niko ipya'e voi oveve ijeho ramo. Upéagui añete voi
oiko araka'e Ñanderu ñe'ẽ va'ýra háramo opa tetã rupi ko yvypýpe.
Mainomby oikuaa ramo oikotaha mba'e porã vy'a guasu pavẽ yvy pype oguãhẽ
voi tekuaty teépy opa tetã rupi mba'e mombe'úvy. Ha ñande johechakáry
katu oikuaa ojehutaha ma'e marandi yvy pype ha'e oho ave chirino rami
oñomongeta rag̃ua ñanderu ete ndive yváypy. Upéagui ñanderu araka'e
ombojaru ha omoñe'ẽngue mainombýpe he'i chupe:

\begin{quote}
Para nossos antepassados, Beija-flor se mostra como mensageiro da
palavra das divindades, pois seu voo é incrivelmente veloz. Por isso,
ele é considerado o mensageiro de notícias de Ñanderu, nossa divindade
máxima, para todas as nações da terra. Se Beija-flor sabe que vão
acontecer coisas boas, grande festa, leva a notícia para todas as
comunidades. E os nossos \emph{johechakáry}\footnote{\emph{Johechakáry}
  pode ser traduzido como ``aquele que tem acesso aos patamares das
  divindades'', já que \emph{johecha} se refere a ``ver'' ou
  ``observar'' as divindades e \emph{kary,} à pessoa única que possui
  este poder. O trânsito pelos patamares onde vivem as divindades é
  garantido a esses \emph{ñanderu}, rezadores, por terem corpo e alma
  puros, tendo ultrapassado o \emph{aguyje}, o estado maturação
  corporal, e alcançado o dom da profecia. Para adquirir uma visão e uma
  audição hiperpotentes, aqueles que se tornam \emph{johechakáry} passam
  por um processo de renovação ou troca da carcaça corporal, que,
  segundo Atanásio Teixeira, implica na retirada de quatro peles
  transparentes de seus olhos e outras quatro de seus ouvidos.} têm
acesso direto às divindades e, quando vai acontecer algo ruim na
comunidade, as suas palavras se transformam num colibri e vão até
Ñanderu Vusu para conversar. Por isso, Ñanderu Vusu brincou com
Beija-flor e assim cantou:
\end{quote}

\begin{verse}
Opy opyryry\\
Mainombya cherehe

Opy opyryry\\
Mainombya cherehe

Vytera teniko\\
Kaguῖ retéra che juka\footnotemark{}\\
Vytera teniko\\
Kaguῖ retéra che juka

Nde yvotyrya, guyra miñaeko\\
Nde juka, Mainomby\\
Nde yvotyrya, guyra miñaeko\\
Nde juka, Mainomby

Vytera teniko\\
Kaguῖ retéra che juka\\
Vytera teniko\\
Kaguῖ retéra che juka
\end{verse}

\footnotetext{O verbo \emph{-juka} pode ser traduzido como ``matar'',
  mas no Guarani também pode adquirir o sentido de ``embebedar-se'',
  empregado nestes versos do \emph{guahu} do Beija-flor.}

\begin{verse}
Gira, gira\\
Beija-flor, para mim

Gira, gira\\
Beija-flor, para mim

Néctar pleno\\
Corpo\footnotemark{} do cauim me embebeda\\
Néctar pleno\\
Corpo do cauim me embebeda

Com seu Néctar, passarinho\\
Você se embebeda, Beija-flor\\
Com seu Néctar, passarinho\\
Você se embebeda, Beija-flor

Néctar pleno\\
O corpo do cauim me embebeda\\
Néctar pleno\\
O corpo do cauim me embebeda
\end{verse}

\footnotetext{\emph{Retéra}, que se refere ao corpo da bebida
  fermentada, conforme o entendimento dos Kaiowa.}

Upéa ñemoñe'ẽngue opyta Mainomby guahu ramo ko yvypýpe.

\begin{quote}
Assim a sua palavra ficou registrada no \emph{guahu} do Beija-flor aqui
na terra.
\end{quote}

\chapter{Kavure guahu / O canto do Caburé}

Guyra kavure ju oikoha rupi ambue guyra'i ha guyra reko poroangu
ndaija'éi voi hese, upéagui ojeahéi oipete pete arã ichupe opepópy. Ha
kavure katu ojeheko repy rag̃ua oiko rari rari arã ichugui ha oñemi arã
yvyra rakã kóta rehe. Upéagui ñande ypy araka'e he'i hese: ``Amondóta
kavure ju yvyra katĩty rehe oñe'ẽ ñe'ẽ rag̃ua ambue tetã rehe ambue guyra
avytekue rupi, upéicha he'i araka'e kavure ju rehe, upéagui ha'e oikoha
rupi ka'aru ha pyhare ypy rag̃ua oñe'e arã he'i: hóĩ, hóĩ, hóĩ, hóĩ,
hóĩ... Upéicha oñe'ẽ kavure ju. Ha iguahu katu omombe'u hembiasakue
rehegua ha upéicha he'i:

\begin{quote}
Caburé\footnote{Caburé é o nome popular de um tipo de coruja do gênero
  \emph{Glaucidium}. Seu nome popular em português, \emph{caburé},
  provém das línguas tupi-guarani.} é um pássaro diferente. Por onde ele
voa, os outros pássaros não gostam de vê-lo voando sossegado, e o
perseguem e agridem. Caburé, para tentar fugir das agressões, às vezes
tenta voar mais rápido, para escapar, ou se esconde atrás dos troncos
das árvores. Por isso, o Ñanderu falava assim: ``Vou mandar Caburé ao
lugar onde só há árvores secas, para cantar a cada anoitecer no galho
seco, no meio dos outros pássaros''. Por isso Caburé canta no entardecer
ou no início da noite, assim: hóĩ, hóĩ, hóĩ, hóĩ, hóĩ.... Assim canta
Caburé. O \emph{guahu} de Caburé narra o seu passado e por onde ele
vive, e como canta em cada entardecer:
\end{quote}

\begin{verse}
Amondo Kavure ju\\
Iñe'ẽ kamavarĩ

Amondo Kavure ju\\
Iñe'ẽ kamavarĩ

Hóĩ, hóĩ, ere Kavure ju\\
Hóĩ, hóĩ, ere Kavure ju

Yvyra katĩ aja katu ereñe'ẽko guyra\\
Yvyra katĩ aja katu ereñe'ẽko guyra
\end{verse}

\begin{verse}
Mandei o Caburé\\
Com seu canto singular

Mandei o Caburé\\
Com seu canto singular

Hóĩ, hóĩ, cante Caburé\\
Hóĩ, hóĩ cante Caburé

No tempo das folhas caídas, canta o pássaro\\
No tempo das folhas caídas, canta o pássaro
\end{verse}

Upéicha kavure guahu oiko yvypýpe.

\begin{quote}
Assim é o canto \emph{guahu} do Caburé.
\end{quote}

\chapter{Mykurẽ guahu / O canto do Gambá}

Mykurẽ oipapa ogueko ha omohembypy ave iguahu rupi. Mykurẽ omboypy ypy
iguahu rupi ikaru oiko ramo guare: upéa rembiasakue yvypórape oheja.
Mykurẽ oipapa heko mba'éichapa ojeheka oiko jave ka'aguy mbyte rupi
omangeko opa yvyrakua rehe ojohu haguã hemiurã, ha upéa heko omoñeysyrũ
iguahu rupi hekóre meme. Upéicha, Mykurẽ omombe'u oguekoypy oguahu rupi.
Upéagui, yvypóra omombe'u joapýri pýri Mykurẽ guahu. Upéagui, Mykurẽ
ñe'ẽ katukue ete guahu rupi ombohysy hekóre guare memete. Upéagui Mykurẽ
guahu he'i:

\begin{quote}
Gambá\footnote{Os Kaiowa conhecem diferentes tipos de gambá -
  \emph{aguara}, \emph{kure'i}, \emph{mbyku}, \emph{mykure},
  \emph{mbyku} - e cada um deles tem características específicas. Este é
  o \emph{guahu} de um desses tipos, o \emph{mykure}, que no português é
  conhecido popularmente pelos nomes saruê e mucura.} conta o seu modo
de vida e compõe a sua história por meio do seu \emph{guahu}. Gambá
criou o seu jeito de viver, e o reconta com o \emph{guahu},
especialmente o seu jeito de comer: esta é a história que Gambá deixou
para os seres humanos da terra. Gambá conta como faz para procurar
comida no meio da mata: ele procura comida no buraco da copa das
árvores. Assim o Gambá conta a sua vida primordial por meio do
\emph{guahu}. Por isso, os seres humanos contam e recontam o
\emph{guahu} do Gambá. Dessa forma, as palavras belas de Gambá são
entoadas nesse canto \emph{guahu} que narra sua vida. O \emph{guahu} de
gambá diz assim:
\end{quote}


\begin{verse}
Yvyra ku'a vei\\
Chekóy karu\\
Chekóy karu

Yvyra ku'a vei\\
Chekóy karu\\
Chekóy karu
\end{verse}

\begin{verse}
Na copa da árvore\\
Eu como assim\\
Eu como assim

Na copa da árvore\\
Eu como assim\\
Eu como assim
\end{verse}

Upéicha Mykurẽ guahu oiko yvy pórape.

\begin{quote}
Assim o canto \emph{guahu} do Gambá segue aqui na terra.
\end{quote}

\chapter{Guasu guahu / O canto do Veado}

Guasu ypy oiko reta araka'e ambue mymba ka'aguy ndive, opamba'e rehe
osareko ha ojesaity. Peteĩ árape Guasu ypy oho oguata mombyry
hekuatýgui. Ha oyvyasa jave ojotopa guyra Yruvu ypy ndive ha reta
oñomongeta teko yvypóra rekorã rehe. Upéi Guasu ypy oho jevy tape rehe
opamba'e rehe osareko ohóvy ijere kuére rehe. Upe tape puku ohojave
otopa jevy Jaguarete ypy: Guasu ohecha jave Jaguarete ypy oguata pya'e
ikyhyjégui ichugui ha upéicha ohasa. Guasu ypy heta oguata rire oguahẽ
peteĩ ambue atyra guasu oikohápe. Upe atyra apytépe Guasu ypy ohecha
mitãkuñarusu, upépy Guasu ypy opyta, omaña kuña porã rehe.

\begin{quote}
No princípio, Veado vivia junto com outros animais do mato, observando
atento as coisas, analisando. Um dia, Veado saiu e caminhou longe do seu
lugar. Durante a caminhada encontrou o pássaro Urubu e conversaram
bastante sobre a vida na terra e como viria a ser. Veado continuou a
caminhada observando atenciosamente as coisas nos seus arredores. Ao
longo do caminho encontrou a Onça: ao avistar a Onça, de medo andou mais
rápido, e assim passou por ela. Depois de muita caminhada, Veado chegou
a um lugar em que se reunia um grande grupo de outros animais. No meio
do grupo, Veado avistou uma moça jovem, e ali ficou parado, observando a
moça bonita.
\end{quote}

Añetehápe Guasu ojehesaity kuña reko rehe ha oñehe'ỹjo, reta ojepy'a
mongeta. Upéa rire Guasu iguahu rupi oipapa kuña porã rehegua.
Oguatahápy Guasu ohecha mitãkuña oguerovy'a, he'i: ``Ijeguaju ha
ijeguaju porã kuña imemby e'ỹ jave''. Upéicha opy'ápype oñemongeta.
Guasu ojesaporavo, ha upéi ohecha mitã kuña rusu imemby ijyva ári,
upéicha he'i Guasu: ``Ha imemby rire katu kuña hova guasu guasu
ijecháy''. Guasu ojepy'a mongeta rire oguahu rupi oguenohẽ oipapa
ojehegua:

\begin{quote}
Veado observou verdadeiramente a vida das mulheres e analisou,
refletindo profundamente. Então Veado, por meio de seu \emph{guahu,}
narra sobre as mulheres belas. Ao passear, Veado avistou uma moça
bonita, ficou risonho e disse: ``Seu enfeite dourado, seu enfeite
dourado é bonito''. Veado observou atentamente, depois viu uma mulher
adulta com seu filho em seus braços e disse: ``Depois de ter filho, a
mulher fica com seu jeito e sua face transformada''. Veado então externa
a sua reflexão, contando por meio de seu \emph{guahu}:
\end{quote}

\begin{verse}
Ijeguaju\\
Ijeguaju porã kuña

Ijeguaju\\
Ijeguaju porã kuña

Imemby rire katu kuña\\
Iguasu rova rova

Imemby rire katu kuña puku\\
Iguasu rova rova
\end{verse}

\begin{verse}
Seu enfeite dourado\\
Seu enfeite dourado, bonita mulher

Seu enfeite dourado\\
Seu enfeite dourado, bonita mulher

Depois de ter filhos, a mulher alta\\
Fica com rosto crescido

Depois de ter filhos, a mulher alta\\
Fica com rosto crescido
\end{verse}

\chapter{Tujuju guahu / O canto do Tuiuiú}

Guyra tujuju oguahu rupi oipapa ojehegua omoherakuã yvypórape oguekoypy.
Guyra Tujuju rembi'u niko pira añónte voi, pira rehe ha'e ojeporeka kuaa
voi yupa guasu rupi okaru rag̃ua, upe rekorã hembypy araka'e omboypy
ichupe. Upéagui ikaruháty haguérupi yupa guasu rupi retave pirapekue
omosãrambi; upéagui tujuju ojehegua omombe'u ha oñemomba'e guasu voi
ha'e añónte imba'e ñyvõ kuaaha. Tujuju ojehe he'i ha'eñónte hetaha
oñyvõha, upéichagui tujuju he'i oguyvyrape eju torombo'e eju torombo'e
pirayvõpy, ha iguahu rupi oñemombe'u ha oipapa ogueko ipira jukaha;
upéicha ha'e oguyvyrape he'i. Tujuju niko itĩ ojekuaa ichupe hu'i ramo
ha upéapy oñyvõ pirápe ojuka ha ho'u rag̃ua, ha upéa va'e gueko tujuju
omomba'e tee voi ha'e añónte ikatupyryha juka kuaaha pirápe. Upéagui
Tujuju omoñesyrũ oguahu ijehegua meme omombe'u.

\begin{quote}
O pássaro Tuiuiú conta sobre seu jeito de viver e se anuncia para os
seres desta terra. O Tuiuiú só come peixe no grande remanso das águas: o
seu jeito de ser foi determinado no tempo do princípio. Dessa forma, por
onde Tuiuiú come, deixa bastante escama de peixe esparramada na beira do
rio; por isso, o pássaro Tuiuiú diz que só ele sabe flechar os peixes.
Pássaro Tuiuiú sempre repete que só ele sabe flechar os peixes, por isso
o Tuiuiú disse assim para seu irmão: vem, meu irmão, que vou te ensinar
a flechar os peixes. Estas palavras são repetidas no canto \emph{guahu}:
que só ele tem habilidade para matar os peixes com a flechada; assim ele
falava para seu irmão. O bico de Tuiuiú para ele é sua flecha, e com
essa flecha ele atira para matar os peixes para comer. Essa atuação
Tuiuiú narra e diz que só ele sabe caçar com facilidades os peixes. Por
isso, o pássaro Tuiuiú por meio de seu \emph{guahu,} canta narrando
sobre o seu jeito de ser.
\end{quote}

\begin{verse}
Eju torombo'e cheryvy\\
Eju torombo'e cheryvy\\
Eju torombo'e cheryvy\\
Eju torombo'e cheryvy

Itaypa vusu rupi\\
Itaypa vusu rupi

Morotῖ, morotῖ\\
Pirapekue, pirapekue\\
Morotῖ, morotῖ\\
Pirapekue, pirapekue

Morotῖ, morotῖ\\
Cherapy kue re rupi\\
Morotῖ, morotῖ\\
Cherapy kue re rupi
\end{verse}

\begin{verse}
Venha que eu te ensino, meu irmão\\
Venha que eu te ensino, meu irmão\\
Venha que eu te ensino, meu irmão\\
Venha que eu te ensino, meu irmão

Sobre as grandes pedras\\
Sobre as grandes pedras

Brancas, brancas\\
Escamas de peixe, escamas de peixe\\
Brancas, brancas\\
Escamas de peixe, escamas de peixe

Brancas, brancas\\
Por onde eu passei\\
Brancas, brancas\\
Por onde eu passei
\end{verse}

\chapter{Araku guahu / O canto da Saracura}

Guyra Araku oiko jojapa oñondive ramo guare, guyra meme oñe'ẽ ha
oñombovi'a guasu. Upéa jave, guyra Araku he'y he'yi ambue avytépy he'i:
``Amombe'úta guahu rupi che jehegua, yvypórape opyta haguã che ñe'ẽ
poravokue''. Upéicha guyra Araku oha'ã iguahu, upéa guahu opyta araku
guahu ramo. Upéagui iguahúpi araku oñehendu uka.

\begin{quote}
No tempo em que o pássaro Saracura ainda vivia falando com todos os
pássaros, só havia pássaros e eles falavam uns com os outros,
alegrando-se grandemente. Neste momento, a Saracura falou no meio dos
outros pássaros de diferentes espécies: ``Vou contar por meio de meu
\emph{guahu}, deixar aos seres desta terra meu canto preferido''. Assim
Saracura cantou seu \emph{guahu,} e esse canto ficou conhecido como de
Saracura. Por isso, por meio de seu \emph{guahu,} Saracura de longe faz
escutar seu canto.
\end{quote}

\begin{verse}
Oñe'ẽ tirikoju\\
Oñe'ẽ tirikoju\\
Oñe'ẽ tirikoju\\
Oñe'ẽ tirikoju

Oñe'ẽ tirikoko\\
Oñe'ẽ tirikoko\\
Oñe'ẽ tirikoko\\
Oñe'ẽ tirikoko
\end{verse}

\begin{verse}
Canta saracura de peito dourado\\
Canta saracura de peito dourado\\
Canta saracura de peito dourado\\
Canta saracura de peito dourado

Canta saracura\\
Canta saracura\\
Canta saracura\\
Canta saracura
\end{verse}

\chapter{Gua'a guahu / O canto da Arara-vermelha}

Guyra gua'a ha'e Ñanderu Vusu rembiguái tee voi, guyra iporavo pyre.
Guyra Gua'a heko namarãi ambue guyra avytépy. Guyra gua'a oguereko
ijeguaka ha iju porã, yvypóra hesa ojejopiágui guyra gua'a ojeguaa guyra
ramo, ijeguaka iju porã porã. Ñanderu yvateguápe katu guyra gua'a reko
marangatu, iñe'ẽ ndaijojái namarãi, guyra katu ete, ha oguejy ramo ko
yvy rehe Ñanderu ñe'ẽ katukue ojekuaa guyra gua'a ramo. Gua'a iguahu
rupi oipapa ojeguaka ha reko marane'ỹ rehegua.

\begin{quote}
A Arara-vermelha é ajudante indispensável para Ñanderu Vusu. Pássaro
arara possui a vida plena, diferente no meio de outros pássaros. A
Arara-vermelha possui um adorno amarelo e muito bonito. Os seres humanos
da terra têm vendagem sobre olhos pássaro arara visto como pássaro, seu
lindo adorno amarelo. Para Ñanderu lá do céu, pássaro arara tem vida
fundamentalmente plena, sua palavra não se compara. Pássaro sagrado,
quando desce na terra, as palavras de Ñanderu aparecem como pássaro
arara. Arara-vermelha, por meio de seu \emph{guahu,} narra sobre seu
adorno e seu modo de vida perfeito.
\end{quote}

\begin{verse}
Jeguakari, jeguakari\\
Se sejá, se sejá

Jeguakari, jeguakari\\
Se sejá, se sejá
\end{verse}

\begin{verse}
Enfeite, enfeite\\
Eu, eu no meu andar, eu, eu no meu andar

Enfeite, enfeite\\
Eu, eu no meu andar, eu, eu no meu andar
\end{verse}

\chapter{Kaninde guahu / O canto da Arara-canindé}

Ñanderu Vusu ko yvypýpe joty oiko jave, guyra kaninde oiko Ñanderu roka
rupi oguata. Upéa heko Kaninde omombe'u ambue guyra ypy kuérype, oguahu
rupi oipapa Ñanderu rymbaha upe guyra. Kaninde ha'e guyra marangatu
ndaijojahái iñe'ẽ, oñendu uka ombovy'a haguã Ñanderúpe. Kaninde oiko
ramo guyra avyte rupi heko ndaikatúi ñambojoja guyra ko yvy rehegua
rehe. Upéagui guyra Kaninde Ñanderu roka rupi oguata va'ekue oipapa heko
oguahu rupi ha oheja yvypórape, ha iñe'ẽ katukue oheja iguahu rupi
omombe'u joapýri pýri heko marane'ỹ. Upéagui Kaninde oguahu rupi he'i:

\begin{quote}
Quando Ñanderu Vusu ainda vivia aqui na terra, Arara-canindé vivia
andando no terreiro de Ñanderu. É este modo de vida que Arara-canindé
contava para os outros pássaros primordiais, por meio de seu
\emph{guahu,} narrando seu passado e falando que a Arara-canindé
pertence ao Ñanderu. Arara-canindé é uma espécie de pássaro sagrado: sua
fala não tem igual, canta para alegrar o Ñanderu. Arara-canindé, quando
vive no meio de outros pássaros, não tem como comparar com os demais.
Por isso, Arara-canindé conta de quando andava no terreiro de Ñanderu
por meio do \emph{guahu}, e revela aos seres humanos suas verdadeiras
palavras, contando sobre essa sua antiga vida de plenitude. Por isso
Canindé, por meio de seu \emph{guahu}, diz:
\end{quote}

\begin{verse}
Apyrũ pyrũ ko konde ruguái rari guyra\\
Apyrũ pyrũ ko konde ruguái rari guyra

Ñanderu roka rupi\\
Ñanderu roka rupi

Apyrũ pyrũ ko konde ruguái rari guyra\\
Apyrũ pyrũ ko konde ruguái rari guyra

Ñanderu roka rupi\\
Ñanderu roka rupi
\end{verse}

\begin{verse}
Pisei, pisei mesmo seu rabo em movimento, pássaro\\
Pisei, pisei mesmo seu rabo em movimento, pássaro

No pátio de Ñanderu\\
No pátio de Ñanderu

Pisei, pisei mesmo seu rabo em movimento, pássaro\\
Pisei, pisei mesmo seu rabo em movimento, pássaro

No pátio de Ñanderu\\
No pátio de Ñanderu
\end{verse}

\chapter{Sapeny guahu / O canto da Gaivota}

Guyra Sapeny oipapa reko oveve oñembiara ara oiko va'ekue araguy rehe
oguahu rupi. Upéa guyra Sapeny omombe'u yvypórape iñe'ẽ katukue rupi,
omoñesyrũ ajoavey vey araguy rehe oiko va'ekue. Guyra Sapeny he'i
araka'e ojoupe: ``Che ave ahejáta che guahu, angerupi yvypóra ojeroeta
ramo omombe'u haguã che guahu''. Upéagui iguahu oipapa reko veve
rehegua:

\begin{quote}
Gaivota\footnote{O \emph{Sapeny} corresponde a uma espécie conhecida
  popularmente em Mato Grosso do Sul como gaivota, taiamã ou
  trinta-réis-grande (\emph{Phaetusa simplex}).} narra o seu passado e
como é o seu jeito de voar debaixo do céu por meio de seu \emph{guahu}.
Gaivota deixou para os seres da terra as palavras verdadeiras sobre o
seu jeito de voar e voar. Gaivota dizia assim: ``Eu também quero deixar
o meu \emph{guahu}, para, no tempo futuro, os seres humanos da terra se
multiplicarem e poderem contar para outros o meu \emph{guahu}''. Por
isso o seu \emph{guahu} conta assim:
\end{quote}

\begin{verse}
Parary joapihi\\
Parary joapi\\
He parary joapihi

Parary joapihi\\
Parary joapihi\\
Parary joapihi

He parary ryjuipy ajoapi\\
He parary ryjuipy ajoapi

Parara joapihi\\
Parara joapi

Yryjuipy ajoapi\\
Yryjuipy ajoapi

Parary ryjuipy ajoapi\\
Parary ryjuipy ajoapi
\end{verse}

\begin{verse}
Grandes águas se encontram\\
Grandes águas se encontram\\
Grandes águas se encontram

Grandes águas se encontram\\
Grandes águas se encontram\\
Grandes águas se encontram

Grandes águas, espumas eu lanço\\
Grandes águas, espumas eu lanço

Grandes águas se encontram\\
Grandes águas se encontram

Espumas eu lanço\\
Espumas eu lanço

Grandes águas, espumas eu lanço\\
Grandes águas, espumas eu lanço
\end{verse}

\chapter{Tapiti guahu / O canto do Tapiti}

Tapiti reko ñeypyrũ oguahu rupi oipapa ojehu va'ekue rekové rehe. Tapiti
omonda araka'e jaguarete rata, upéagui jagarete ho'u so'o pyry. Ha
Tapiti katu oikoha rupi ikyhyje jaguaretégui.

\begin{quote}
Tapiti\footnote{O tapiti (\emph{Sylvilagus brasiliensis}) é um pequeno
  mamífero da família dos coelhos e lebres, nativo do continente
  americano, também conhecido como coelho brasileiro.} narra a origem de
seu modo de vida por meio de seu \emph{guahu}. No passado, Tapiti foi o
responsável por furtar o fogo, que era da Onça. Por isso, hoje, Onça
come carne crua. Tapiti, por sua vez, vive até hoje com medo da Onça.
\end{quote}

Upe Tapiti ha Jaguarete rehegua ñemombe'upy hysy voi ojehu va'ekue.
Tapiti omangeko oikóvy araka'e Jaguarete rata rehe, upe jave heta ary
jehekove mongeta ijupe, mba'éichapa Jaguarete rata omondáta ichugui.
Jaguarete upe mba'e vai oñandu opy'apýpe ojererahaseha ichugui hata;
upemarãmo, Jaguarete ohenói Kururúpe ha Ñakyrãme oñangareko haguã
hataypy rehe oho jave ojeheka ka'guy rehe. Jaguarete ombohekorã Kururúpe
opyta haguã tataypýpe ha Ñakyrã katu opyta haguã yvate yvyra ku'ápe.
Jaguarete omomarandu ha he'i upéicha Kururúpe: ``Erehecha ohu ramo
ko'ápy mondaháry, pya'épy eñatõi kuri Ñakyrãme, ha Ñakyrã aekatu
osapukái haguã chévy. Ahendu ramo ñakyrã sapukái pya'épy, ajúta ahepy
haguã cherata''. Upeicha Jaguarete omomarandu Kururúpe ha Ñakyrãme. Upéi
Jaguarete oho ka'aguy rehe ojeheka rembi'urã rehe, ha Kururu opyta
oguapy tataypýpe, upehágui rete kane'õ ha oke. Ñakyrã katu oĩme yvyra
ku'ápe ojapysaka kururu rehe. Tapiti ohecha Kururu oke, mbeguekatúpe
ojemboja ha ogueroña jaguarete rata. Kururu opáy Tapiti ogueraháma rire
Jaguarete rata, Kururu oñemondýi ha ñe'ẽ hatãme oñatõi Ñakyrãme osapukái
haguã Jaguaretépe. Ñakyrã Jaguarete oheko mbo'e haguéicha hatã voi
osapukái ohendu haguã. Jaguarete ohendu Ñakyrã sapukái ou pya'épy ohecha
hataypy ha ndoikovéima. Upemaramo, Jaguarete oporandu Kururúpe mávapa
ogueraha hata, Kururu ndoikuaái mávapa mondaháry. Jaguarete ipochy
etereígui omombo Kururúpe ypugue mbytépy ojuka haguã yrupive. Kururu y
mbytépy ho'a jave ovy'a oikove hague rehe. Upéi Jaguarete ohenõi
ñakyrãme, oporandu ichupe mavápa okueraha hata ichugui. Ñakyrã ndoikuaái
ave mondaháry, ha Jaguarete ipochy ave Ñakyrã ndive ha oity yvýpe opyrũ
hi'ári ojapypy yvyguýpe ichupe. Tapiti katu Jaguarete rata omonda rire
oiko asy ichugui.

\begin{quote}
Sobre Tapiti e Onça, as histórias seguem uma longa sequência de
acontecimentos do passado. Tapiti vivia vigiando o fogo da Onça e
passava vários dias refletindo como fazer para roubar o fogo da Onça.
Onça pressentiu essas coisas ruins no seu pensamento: alguém cobiçava
levar seu fogo embora. Nessa situação, Onça chamou Sapo e Cigarra para
cuidar do fogo na sua ausência, quando fosse para o mato. Onça instruiu
Sapo para ficar ao lado do fogo e a Cigarra a ficar no alto, na copa da
árvore. Onça ensinou e disse para Sapo: ``Caso você perceba o ladrão
vindo aqui, rapidamente avise Cigarra, e a Cigarra vai gritar para mim.
Assim que ouvir o grito da Cigarra, vou vir rápido para salvar o fogo''.
Assim a Onça ensinou o Sapo e a Cigarra. Então Onça foi para o mato à
procura de comida, Sapo ficou sentado à beira do fogo e, depois, com
corpo cansado, dormiu. Cigarra estava na copa da árvore atenta ao aviso
de sapo. Tapiti viu que Sapo já estava dormindo, veio devagar,
aproximou-se e levou o fogo da Onça. Sapo acordou depois que Tapiti
levou o fogo e avisou a Cigarra, assustado e falando alto, para gritar a
Onça. Cigarra seguiu a instrução dada pela Onça e gritou alto mesmo,
para ser ouvida. Onça ouviu o grito da Cigarra, veio rapidamente e viu
que seu fogo não estava mais no lugar. Nessa situação, Onça perguntou
para Sapo quem havia levado o fogo, mas Sapo não sabia dizer quem era o
ladrão. Onça ficou muito brava, pegou o Sapo e o arremessou no meio da
lagoa, com a intenção de matá-lo. Ao cair no meio da lagoa, Sapo ficou
muito alegre por ter sobrevivido. Depois, Onça chamou Cigarra e
perguntou para ela quem tinha levado seu fogo. Cigarra também não sabia
quem era o ladrão, então Onça ficou brava com ela, derrubou-a no chão,
pisou em cima e enterrou-a na terra. Tapiti não tinha mais sossego
depois que levou o fogo da Onça.
\end{quote}

Upéagui peteĩ ára Tapiti oñomi haguã opypore Pa'i Kuarápe oho oporandu
mba'éichapa Jaguarete rape ypýgui oñomi va'erã opypore. Pa'i Kuara
iñarandu rupi ojavoa ipy ha ipo mandijúpy; upéagui ohasa ramo jepe
jaguarete rape ypy rupi ndojekuaái tapiti pypore. Upéa ogueko Tapiti
oguahu rupi oipapa ha oñembohory Jaguarete rehe. Ha upéicha he'i:

\begin{quote}
Tapiti foi, então, perguntar para o Sol o que poderia fazer para
esconder seu rastro ao passar ao lado do trilheiro da Onça. O Sol, com
muita sabedoria, enrolou a pata e a perna de Tapiti com algodão; com
essa vestimenta, ao passar perto do trilheiro da Onça, ela não
conseguiria perceber o Tapiti. É isso que o Tapiti conta por meio de seu
\emph{guahu}, que também canta como forma de provocar a Onça. E assim
diz:
\end{quote}

\begin{verse}
Añomi, añomi che pypore ava rapeypy peỹ\\
Añomi, añomi che pypore ava rapeypy peỹ

Chepy pokuaa pota re'ỹ\\
Chepy pokuaa pota re'ỹ

Añomi añomi che pypore ava rapeypy peỹ\\
Añomi añomi che pypore ava rapeypy peỹ

Chepy pokuaa pota re'ỹ\\
Chepy pokuaa pota re'ỹ
\end{verse}

\begin{verse}
Escondi, escondi meu rastro ao lado do trilheiro do meu inimigo\\
Escondi, escondi meu rastro ao lado do trilheiro do meu inimigo

Meu pé não quer acostumar\\
Meu pé não quer acostumar

Escondi, escondi meu rastro ao lado do trilheiro do meu inimigo\\
Escondi, escondi meu rastro ao lado do trilheiro do meu inimigo

Meu pé não quer acostumar\\
Meu pé não quer acostumar
\end{verse}


