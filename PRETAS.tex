\textbf{Cantos dos animais primordiais} apresenta 26 histórias de aves e outros animais da
mata, acompanhados pelos cantos \textit{guahu} que cantam sua história
desde o princípio dos tempos. Esses \textit{guahu} fazem parte de um
conjunto maior de \textit{cantos-rezas-danças}, conforme dominado pelo xamã e autor, que podemos traduzir como ``cantos míticos''. As
narrativas e explicações que acompanham os cantos foram elaboradas por
Izaque João, a partir de falas e orientações de Atanásio Teixeira ao
longo dos últimos seis anos. Os processos de seleção, transcrição e
tradução para esta edição bilíngue também foram feitos em diálogo com o
xamã e as versões em português dos textos e cantos \textit{guahu} são um
exercício de aproximação a suas belas palavras.

\textbf{Ava Ñomoandyja Atanásio Teixeira} (1922) é um dos mais importantes \textit{ñanderu} ou ``rezador'' do povo Kaiowá em atividade. Nascido em 1922, Ataná é chamado de \textit{ñamoῖ}, avô, por lideranças e rezadores de diferentes comunidades kaiowá, pelos quais é reconhecido como mestre. É um dos precursores dos \textit{jeroky guasu}, as ``grandes danças'' dos anos 1980, e do movimento histórico pela recuperação dos territórios kaiowá e guarani em Mato Grosso do Sul, a \textit{Aty Guasu}, ``grande reunião'', além de ser reconhecido como um grande xamã também pelos Guarani. O prestígio de Atanásio está, entre outros motivos, ligado ao fato de dominar as mais variadas técnicas ligadas ao xamanismo kaiowá: os \textit{ñembo'e}, fórmulas verbais de proteção pessoal ou coletiva; os \textit{mborahei} e \textit{guahu}, cantos coletivos ligados aos rituais; os diversos tipos de gestos conhecidos como \textit{jehovasa} --- que podem ser utilizados para influenciar as condições climáticas, desviando tempestades, por exemplo; para curar doenças físicas e espirituais; para garantir a sanidade das lavouras e colheitas etc.

\textbf{Izaque João}, do povo Kaiowá, é professor e pesquisador dedicado ao estudo dos \textit{cantos-rezas} e conhecimentos tradicionais dos povos Kaiowá e Guarani. Doutorando em
Antropologia na Universidade de São Paulo (\textsc{usp}), é também mestre em História pela
Universidade Federal da Grande Dourados (\textsc{ufgd}). Vive atualmente na Reserva
Indígena de Dourados (\textsc{ms}), onde coordena o Magistério Indígena Ára Verá. Já
coordenou pesquisas para organizações como o Museu do Índio (Funai) e o Fundo
Brasil de Direitos Humanos. É correalizador do documentário \textit{Monocultura da fé} (2018).

\textbf{Coleção Mundo Indígena} reúne materiais produzidos com pensadores de diferentes povos indígenas e pessoas que pesquisam, trabalham ou lutam pela garantia de seus direitos. Os livros foram feitos para serem utilizados pelas comunidades envolvidas na sua produção, e por isso uma parte significativa das obras é bilíngue. Esperamos divulgar a imensa diversidade linguística dos povos indígenas no Brasil, que compreende mais de 150 línguas pertencentes a mais de trinta famílias linguísticas.



