%\newcommand{\linhalayout}[2]{{\tiny\textbf{#1}\quad#2\par}}
\newcommand{\linha}[2]{\ifdef{#2}{\linhalayout{#1}{#2}}{}}

\begingroup\tiny
\parindent=0cm
\thispagestyle{empty}

\textbf{edição brasileira\,©}\quad			 {Hedra \the\year}\\
\textbf{organização e tradução\,©}\quad	 {Izaque João}\\
\textbf{coorganização\,©}\quad			 	 	 	 {Spensy Pimentel e Tatiane Klein}\\

\textbf{coordenação da coleção}\quad		 {Luísa Valentini}\\
\textbf{edição}\quad			 			 {Suzana Salama}\\
\textbf{assistência editorial}\quad			 {Paulo Henrique Pompermaier}\\
\textbf{revisão do guarani}\quad			 {Arnulfo Morínigo Caballero}\\
\textbf{revisão do português}\quad			 {Spensy Pimentel, Tatiane Klein e Luísa Valentini}\\
\textbf{capa}\quad			 			{Lucas Kröeff}\\

\textbf{\textsc{isbn}}\quad			 				 {978-65-89705-30-7}\smallskip

\vfill

\begin{minipage}{7cm}
\textbf{Dados Internacionais de Catalogação na Publicação (\textsc{cip})\\
(Câmara Brasileira do Livro, \textsc{sp}, Brasil)}

\textbf{\hrule}\smallskip

Teixeira, Ava Ñomoandyja Atanásio\\

\textit{Cantos dos animais primordiais}. Ava Ñomoandyja Atanásio Teixeira; organização e tradução de Izaque João. 1.\,ed. São Paulo, \textsc{sp}: Hedra, 2025.\\


\textsc{isbn} 978-65-89705-30-7\\

1.\,Conto. 2.\,Literatura brasileira. \textsc{i.}\,Teixeira, Ava Ñomoandyja Atanásio. \textsc{ii}.\.Izaque João. \textsc{iii}.\,Título.

\hfill \textsc{cdd}: 869.93

\textbf{\hrule}\smallskip

\textbf{Elaborado por Janaina Ramos (\textsc{crb} 8/\,9166)}\\

\textbf{Índices para catálogo sistemático:}\\
\textsc{i.}\,Conto : Literatura brasileira

\end{minipage}

\vfill

\textit{Grafia atualizada segundo o Acordo Ortográfico da Língua\\
Portuguesa de 1990, em vigor no Brasil desde 2009.}\\

\textit{Direitos reservados em língua\\
portuguesa somente para o Brasil}\\

\textsc{editora hedra ltda.}\\
Rua Sete de Abril, 235, cj.\,102\\
01043--000 São Paulo \textsc{sp} Brasil\\
Telefone/Fax +55 11 3097 8304\\\smallskip
editora@hedra.com.br\\
www.hedra.com.br\\
\bigskip

Foi feito o depósito legal.

\endgroup
\pagebreak